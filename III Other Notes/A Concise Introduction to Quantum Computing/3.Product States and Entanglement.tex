\begin{definition}\label{defn2}
  A quantum state $|\psi\rangle \in\left(\mathbb{C}^2\right)^{\otimes q}$ is a \textbf{product state} if it can be expressed as a tensor product $\left|\psi_1\right\rangle \otimes \cdots \otimes\left|\psi_q\right\rangle$ of $q$ 1-qubit states. Otherwise, it is \textbf{entangled state }.
\end{definition}

Why would we need this definition?

If $|\psi\rangle$ belongs to $\left(\mathbb{C}^2\right)^{\otimes q}=\underbrace{\mathbb{C}^2 \otimes \cdots \otimes \mathbb{C}^2}_{q \text { times }}$, then it \textit{seems} that there exist q vectors, $\left|\psi_k\right\rangle$ in $\mathbb{C}^2$ for $k=1,\dots,q$, such that they satisfy the relation $$|\psi\rangle=\left|\psi_1\right\rangle \otimes \cdots \otimes\left|\psi_q\right\rangle.$$ Just like when we understand the Cartesian product $\mathbb{C}^2 \times \cdots \times \mathbb{C}^2$. However, this is completely wrong!
Is it because we require them all in Definition \ref{defn2} to be quantum states, i.e., unit length constraints? It's still wrong! \textit{What is more important is the definition of tensor product.}


\begin{remark}[Mention Again:Definition 1.1 of tensor product]
    Given two vector spaces $V$ and $W$ with bases $e_1, \ldots, e_m$ and $f_1, \ldots, f_n$, respectively, the tensor product $V \otimes W$ is another vector space of dimension $m n$. The tensor product space is equipped with a bilinear operation $$\otimes: V \times W \rightarrow V \otimes W.$$ The vector space $V \otimes W$ has basis $e_i \otimes f_j, \forall i=1, \ldots, m, j=1, \ldots, n$.
    
    \textbf{Warning!} We should notice that:
    \begin{enumerate}
        \item Unlike the linear function, the range of bilinear function $\otimes$ is not in general a vector space. That is,
            \begin{equation*}
                \operatorname{Range} \otimes:= \left\{x \otimes y :x \in V, y \in W \right\} \subsetneq (V \otimes W)
            \end{equation*} 
            is not a subspace of $V \otimes W$.
        \item However, $\operatorname{Range} \otimes$ contains the all $mn$ basis of $V \otimes W$. In fact, $V \otimes W$ is generated by $\operatorname{Range} \otimes$, or say, $$\operatorname{Span}(\operatorname{Range} \otimes)=V \otimes W.$$
    \end{enumerate}
\end{remark}

\begin{example}[$\operatorname{Range} \otimes$ is not a vector space]
    As an example, let $V = W$ be $\mathbb{C}^2$. Taking
    \begin{equation}
    |\psi\rangle=\left[\begin{array}{l}
    \alpha_0 \\
    \alpha_1
    \end{array}\right] \in \mathbb{C}^2, \quad
    |\phi\rangle=\left[\begin{array}{l}
    \beta_0 \\
    \beta_1
    \end{array}\right] \in \mathbb{C}^2,
    \end{equation}
    then we have
    \begin{equation}\label{eq_tensor}
        |\psi\rangle \otimes|\phi\rangle=\left[\begin{array}{l}
        \alpha_0\beta_0 \\
        \alpha_0\beta_1 \\
        \alpha_1\beta_0 \\
        \alpha_1\beta_1
        \end{array}\right] \in\left(\mathbb{C}^2\right)^{\otimes 2} \equiv \mathbb{C}^4.
    \end{equation}
    We observed that
    \begin{equation}
    \left(\begin{array}{l}
    2 \\
    2 \\
    1 \\
    1
    \end{array}\right)=\left(\begin{array}{l}
    2 \\
    1
    \end{array}\right) \otimes\left(\begin{array}{l}
    1 \\
    1
    \end{array}\right)
    , \quad
    \left(\begin{array}{l}
    1 \\
    0 \\
    1 \\
    0
    \end{array}\right)=\left(\begin{array}{l}
    1 \\
    1
    \end{array}\right) \otimes\left(\begin{array}{l}
    1 \\
    0
    \end{array}\right).
    \end{equation}
    We can prove that their difference 
    \begin{equation}\label{vector}
    \left(\begin{array}{l}
    2 \\
    2 \\
    1 \\
    1
    \end{array}\right)-\left(\begin{array}{l}
    1 \\
    0 \\
    1 \\
    0
    \end{array}\right)=\left(\begin{array}{l}
    1 \\
    2 \\
    0 \\
    1
    \end{array}\right)
    \end{equation}
    is not in the set $\operatorname{Range} \otimes$ thus proving that it is not a vector space.
    Because if (\ref{vector}) has the form (\ref{eq_tensor}), then we must find the solutions of the following four equations
    \begin{equation}
        \left\{
        \begin{aligned}
        & \alpha_0 \beta_0=1 \\
        & \alpha_0 \beta_1=2 \\
        & \alpha_1 \beta_0=0 \\
        & \alpha_1 \beta_1=1
        \end{aligned}
        \right.
    \end{equation}
    which are not possible.

\end{example}


\begin{example}[Another expression for $\operatorname{Range} \otimes$]
    Continue with the above example. We can prove that the following equality holds. This clearly illustrates its non-linearity.
        \begin{align}
        \operatorname{Range} \otimes & :=\left\{|\psi\rangle \otimes|\phi\rangle=\left[\begin{array}{l}
        \alpha_0\beta_0 \\
        \alpha_0\beta_1 \\
        \alpha_1\beta_0 \\
        \alpha_1\beta_1
        \end{array}\right] \in \mathbb{C}^4:
        |\psi\rangle=\left[\begin{array}{l}
        \alpha_0 \\
        \alpha_1
        \end{array}\right] \in \mathbb{C}^2,
        |\phi\rangle=\left[\begin{array}{l}
        \beta_0 \\
        \beta_1
        \end{array}\right] \in \mathbb{C}^2
        \right\} \\
        & =\left\{|\xi\rangle=\left[\begin{array}{l}\gamma_{00} \\ \gamma_{01} \\ \gamma_{10} \\ \gamma_{11} \end{array}\right] \in \mathbb{\mathbb { C }}^{4}: \gamma_{00} \gamma_{11}=\gamma_{01} \gamma_{10}\right\}. \label{set2}
        \end{align}
    \begin{proof}
        The ``only if'' part ($\Longrightarrow$) is trivial, so we omit it. 
        
        For ``if'' part ($\Longleftarrow$), we assume that the given vector $|\xi\rangle$ in set (\ref{set2}) is nonzero, otherwise the proof is trivial. Let $\| |\xi\rangle \| =: \tau >0$. Then $\left|\gamma_{00}\right|^2+\left|\gamma_{01}\right|^2+\left|\gamma_{10}\right|^2+\left|\gamma_{11}\right|^2 = \tau^2.$
        To complete the proof, we only need to construct four complex numbers $\alpha_0, \alpha_1, \beta_0, \beta_1$ such that they satisfy the following four equations:
        \begin{equation}
            \left\{
            \begin{aligned}
            & \alpha_0 \beta_0=\gamma_{00} \\
            & \alpha_0 \beta_1=\gamma_{01} \\
            & \alpha_1 \beta_0=\gamma_{10} \\
            & \alpha_1 \beta_1=\gamma_{11}
            \end{aligned}
            \right.
    \end{equation}
    Let $\theta_{00}, \theta_{01}, \theta_{10}, \theta_{11}$ be the phases (or angles or arguments) and $\left|\gamma_{00}|,\right| \gamma_{01}|,| \gamma_{10}|,| \gamma_{11} |$ be the magnitudes (or modulus or absolute values) of $\gamma_{00}, \gamma_{01}, \gamma_{10}, \gamma_{11}$.
    \footnote{Hence, we can write those complex numbers in polar coordinates:
	$
	\gamma_{00}=\left|\gamma_{00}\right| e^{i \theta_{00}}, \gamma_{01}=\left|\gamma_{01}\right| e^{i \theta_{01}}, \gamma_{10}=\left|\gamma_{10}\right| e^{i \theta_{10}}, \gamma_{11}=\left|\gamma_{11}\right| e^{i \theta_{11}}.
	$}
    Notice that the condition $\gamma_{00} \gamma_{11}=\gamma_{01} \gamma_{10}$ implies
    \begin{align}
    \left|\gamma_{00}\right|^2\left|\gamma_{11}\right|^2 & =\left|\gamma_{01}\right|^2\left|\gamma_{10}\right|^2, \label{eq17} \\
    \theta_{00}+\theta_{11} & =\theta_{01}+\theta_{10}. \label{eq18} 
    \end{align}
    Then we can write 
    \begin{equation}
    \begin{aligned}
    \left|\gamma_{00}\right| & =\sqrt{\left|\gamma_{00}\right|^2}
    =\sqrt{
    \left|\gamma_{00}\right|^2
    \frac{\left(\left|\gamma_{00}\right|^2+\left|\gamma_{01}\right|^2+\left|\gamma_{10}\right|^2+\left|\gamma_{11}\right|^2\right)}{\tau^2}
    }
    \\
    & =\sqrt{
    \frac{\left|\gamma_{00}\right|^4+\left|\gamma_{00}\right|^2\left|\gamma_{01}\right|^2+\left|\gamma_{00}\right|^2\left|\gamma_{10}\right|^2+\left|\gamma_{00}\right|^2\left|\gamma_{11}\right|^2}{\tau^2}
    } \\
    & =\sqrt{
    \frac{\left|\gamma_{00}\right|^4+\left|\gamma_{00}\right|^2\left|\gamma_{01}\right|^2+\left|\gamma_{00}\right|^2\left|\gamma_{10}\right|^2+\left|\gamma_{01}\right|^2\left|\gamma_{10}\right|^2}{\tau^2}
    }  \text{ (By (\ref{eq17}) } \\
    & =
    \underbrace{
    \sqrt{\frac{\left|\gamma_{00}\right|^2+\left|\gamma_{01}\right|^2}{\tau}}
    }_{\left|\alpha_0\right|:=} 
    \underbrace{
    \sqrt{\frac{\left|\gamma_{00}\right|^2+\left|\gamma_{10}\right|^2}{\tau}}
    }_{\left|\beta_0\right|:=} 
    \end{aligned}
    \end{equation}
    and similarly for the other coefficients:
    \begin{equation}
    \begin{aligned}
    & \left|\gamma_{01}\right|
    =\underbrace{
    \sqrt{\frac{\left|\gamma_{00}\right|^2+\left|\gamma_{01}\right|^2}{\tau}}
    }_{\left|\alpha_0\right|}
    \underbrace{
    \sqrt{\frac{\left|\gamma_{01}\right|^2+\left|\gamma_{11}\right|^2}{\tau}}
    }_{\left|\beta_1\right|:=}, \\
    & \left|\gamma_{10}\right|
    =\underbrace{
    \sqrt{\frac{\left|\gamma_{10}\right|^2+\left|\gamma_{11}\right|^2}{\tau}}
    }_{\left|\alpha_1\right|:=}
    \underbrace{
    \sqrt{\frac{\left|\gamma_{00}\right|^2+\left|\gamma_{10}\right|^2}{\tau}}
    }_{\left|\beta_0\right|}, \\
    & \left|\gamma_{11}\right|
    =\underbrace{
    \sqrt{\frac{\left|\gamma_{10}\right|^2+\left|\gamma_{11}\right|^2}{\tau}}
    }_{\left|\alpha_1\right|}
    \underbrace{
    \sqrt{\frac{\left|\gamma_{01}\right|^2+\left|\gamma_{11}\right|^2}{\tau}}
    }_{\left|\beta_1\right|}. \\
    &
    \end{aligned}
    \end{equation}
    To fully define the coefficients $\alpha_0, \alpha_1, \beta_0, \beta_1$ we must determine their phases. We can assign:
	\begin{equation}
    \alpha_0=e^{i \theta_{00}}\left|\alpha_0\right|, \quad \alpha_1=e^{i \theta_{10}}\left|\alpha_1\right|, \quad \beta_0=\left|\beta_0\right|, \quad \beta_1=e^{i\left(\theta_{01}-\theta_{00}\right)}\left|\beta_1\right| .
    \end{equation}
    Moreover, we have 
    $
    \left|\alpha_0\right|^2+\left|\alpha_1\right|^2=\left|\beta_0\right|^2+\left|\beta_1\right|^2=\tau,
    $
    and
    \begin{equation}\label{tau}
        \left\| | \psi\rangle \right\|  = \left\|  |\phi\rangle \right\|  = \sqrt{\tau}.
    \end{equation}

    In summary, for any given nonzero $|\xi\rangle=\left[\begin{array}{c}\gamma_{00} \\ \gamma_{01} \\ \gamma_{10} \\ \gamma_{11}\end{array}\right] \in \mathbb{C}^4 $ with $ \gamma_{00} \gamma_{11}=\gamma_{01} \gamma_{10}$, we can select $|\psi\rangle \in \mathbb{C}^2,|\phi\rangle \in \mathbb{C}^2$ such that
    \begin{equation}
        |\xi\rangle=|\psi\rangle \otimes|\phi\rangle.
    \end{equation}
    Moreover, $\||\psi\rangle\|=\||\phi\rangle \|=\sqrt{\||\xi\rangle \|}$.
    
    \end{proof}
\end{example}

\begin{definition}[Mention Again]
  A quantum state $|\psi\rangle \in\left(\mathbb{C}^2\right)^{\otimes q}$ is a \textbf{product state} if it can be expressed as a tensor product $\left|\psi_1\right\rangle \otimes \cdots \otimes\left|\psi_q\right\rangle$ of $q$ 1-qubit states. Otherwise, it is \textbf{entangled state}.
\end{definition}

NOTE: None of our previous discussions dealt with the quantum state, i.e., the unit length constraint.

We want to investigate whether the quantum states that are representable on $q$ qubits are simply the tensor product of $q$ 1-qubit states.

\begin{example}
        Based on the previous example,
        we give the sufficient and necessary conditions for the product state when $q = 2$. We claim that
        $$\left\{|\psi\rangle \otimes|\phi\rangle \in \mathbb{C}^4:
        \text{ quantum state } |\psi\rangle \in \mathbb{C}^2,
        |\phi\rangle \in \mathbb{C}^2
        \right\} 
        =\left\{\text{ quantum state } |\xi\rangle \in \mathbb{\mathbb { C }}^{4}: \gamma_{00} \gamma_{11}=\gamma_{01} \gamma_{10}\right\}.
        $$
        \begin{proof}
            It comes from the Corollary \ref{cor1} and (\ref{tau}) with $\tau=1$.
        \end{proof} 
\end{example}


\begin{example}
   The situation $\gamma_{00} \gamma_{11}=\gamma_{01} \gamma_{10}$, to verify if the coefficients of a 2-qubit state $|\xi\rangle$ can be expressed as a tensor product of two 1-qubit states, can also be written in matrix form:
	$$
	\left(\begin{array}{ll}
		\gamma_{00} & \gamma_{01} \\
		\gamma_{10} & \gamma_{11}
	\end{array}\right)
	$$
	Then, $|\xi\rangle$ is a tensor product of two 1-qubit states if and only if this matrix has rank 1.

    \begin{itemize}
        \item The rank of $\left(\begin{array}{cc}\frac{3}{4} & \frac{\sqrt{3}}{4} \\ \frac{\sqrt{3}}{4} & \frac{1}{4}\end{array}\right)$ is 1, so $|\psi\rangle =\frac{3}{4}|00\rangle+\frac{\sqrt{3}}{4}|01\rangle+\frac{\sqrt{3}}{4}|10\rangle+\frac{1}{4}|00\rangle$ is a product state and it can be decomposed to $\left(\frac{\sqrt{3}}{2}|0\rangle+\frac{1}{2}|1\rangle\right)\otimes\left(\frac{\sqrt{3}}{2}|0\rangle+\frac{1}{2}|1\rangle\right)$.
        \item The rank of $\left(\begin{array}{cc}\frac{\sqrt{2}}{2} & 0 \\ 0 & \frac{\sqrt{2}}{2}\end{array}\right)$ is 2, so $|\phi\rangle=\frac{\sqrt{2}}{2}|00\rangle+\frac{\sqrt{2}}{2}|11\rangle$ is an entangled state.
    \end{itemize}
\end{example}