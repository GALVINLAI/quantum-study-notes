Consider the standard form of Linear Programming(LP) optimization problem

\begin{equation}\label{primal problem}
\begin{aligned}
& \min _x c^{\top} x \\
& \text { (P) } \quad \text { s.t. } A x=b \\
x &\geq 0,
\end{aligned}
\end{equation}
where $A\in \mathbb{R}^{m \times n}$ with $\operatorname{rank}(A)=m$, $ b \in \mathbb{R}^m$ and $x \in \mathbb{R}^n$. Then consider the dual problem of (P)

\begin{equation}\label{dual problem}
\begin{aligned}
\max _{y, s} b^{\top} y & \\
\text { (D) } \quad \text { s.t. } A^{\top} y+s & =c \\
s & \geq 0,
\end{aligned}
\end{equation}
where $A\in \mathbb{R}^{m \times n}$ with $\operatorname{rank}(A)=m$, $y \in \mathbb{R}^m$ and $s, c \in \mathbb{R}^n$. If $x$ is the optimal solution to the (P) and $(y, s)$ is the optimal solution to the (D), then the following four equations are satisfied.

\begin{equation}
\left\{\begin{array}{l}
A x=b \\
A^T y+s=c \\
X s=0 \\
(x, s) \geq 0
\end{array}\right.
\end{equation}
The Solution $(x, y, s)$ that satisfies (\ref{primal dual problem}) is called an \alert{optimal solution}. Under the condition of $(x, s) \geq 0$, the equation $Xs = 0$ is \alert{equivalent} to $x^{\top}s = 0$. So all optimal solutions, if exist, belong to the set $\mathcal{P} \mathcal{D}^*$, which is defined as
$$
\mathcal{PD}^*=\left\{(x, y, s) \in \mathbb{R}^{n+m+n}: A x=b, A^{\top} y+s=c, x^{\top} s=0,(x, s) \geq 0\right\}.
$$
The elements of the set 
$$
F_{P D}=\left\{(x, y, s) \mid A x=b, A^T y+s=c, (x,s) \geq 0 \right\}
$$
are called the \alert{feasible solutions}. The elements of the set 
$$
F_{P D}^0=\left\{(x, y, s) \mid A x=b, A^T y+s=c, (x,s) > 0 \right\}
$$
are called the \alert{feasible interior points}. Given the parameters $\mu>0$, the \alert{analytic center} of (\ref{primal dual problem}) is the solution of the system

\begin{equation}\label{center path system}
    \left\{
    \begin{aligned}
        & A x=b, \\
        & A^T y+s=c, \\
        & X s=\mu e ,\\
        & (x,s) \geq 0.
    \end{aligned}
    \right.
\end{equation}

If there exists a feasible interior point, then for any $\mu > 0$ there is only one analytic center. The set of analytic centers of (\ref{center path system})
\begin{align*}
    \mathcal{CP} = \{ &(x(\mu), y(\mu), s(\mu)) \in \mathbb{R}^{n+m+n}: A x = b, A^{\top} y + s = c, \\
    &X s=\mu e, (x,s) \geq 0, \mu > 0\}.
\end{align*}
is called \alert{center path}.

The basic idea of center path method is described as below. As $\mu \rightarrow 0$, the points $(x(\mu), y(\mu), s(\mu))$ on the $\mathcal{CP}$ converge to a point $(x^*, y^*, s^*)$ on $\mathcal{PD^*}$. Therefore, for sufficiently small $\mu > 0$, the linear programming problem can be solved by finding the \alert{approximate point} of the analytic center of the primal-dual problem.

We numerically find the optimal solution to the primal dual problem, but it is difficult to find a solution that strictly satisfies the equality constraint. So we need to change the equation constraints. For sufficiently small numbers $\epsilon_P>0, \epsilon_D>0, \epsilon>0$, it is assumed that it is sufficient to find a solution $(x,y,s)$ that satisfies 
$$
\| A  x- b\| \leq \epsilon_P,\left\| A^T  y+ s- c\right\| \leq \epsilon_D,  x^T  s \leq \epsilon.
$$


If $(x,s)>0$, the point $(x,y,s)$ is called the interior point of . The elements of the set
$$
I_{PD}^0 = \{(x,y,s) \mid x\notin F_{PD}^{0}, (x,s)>0\}
$$
are called infeasible interior points. The most important feature of the IIPM is that any feasible or infeasible interior point can be chosen as the initial point.

