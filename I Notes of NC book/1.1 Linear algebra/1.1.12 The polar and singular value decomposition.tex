The polar and singular value decompositions are useful ways of breaking linear operators up into simpler parts. In particular, these decompositions allow us to break general linear operators up into products of unitary operators and positive operators. 

% While we don't understand the structure of general linear operators terribly well, we do understand unitary operators and positive operators in quite some detail. The polar and singular value decompositions allow us to apply this understanding to better understand general linear operators.

\begin{theorem}
Theorem 2.3: (Polar decomposition) Let $A$ be a linear operator on a vector space $V$. Then there exists unitary $U$ and positive operators $J$ and $K$ such that
$$
A=U J=K U
$$
where the unique positive operators $J$ and $K$ satisfying these equations are defined by $J \equiv \sqrt{A^{\dagger} A}$ and $K \equiv \sqrt{A A^{\dagger}}$. Moreover, if $A$ is invertible then $U$ is unique.
\end{theorem}

We call the expression $A=U J$ the left polar decomposition of $A$, and $A=K U$ the right polar decomposition of $A$. Most often, we'll omit the 'right' or 'left' nomenclature, and use the term 'polar decomposition' for both expressions, with context indicating which is meant.

\begin{proof}
    $J \equiv \sqrt{A^{\dagger} A}$ is a positive operator, so it can be given a spectral decomposition, $J=$ $\sum_{i} \lambda_{i}|i\rangle\langle i|\left(\lambda_{i} \geq 0\right)$. Define $\left|\psi_{i}\right\rangle \equiv A|i\rangle$. From the definition, we see that $\left\langle\psi_{i} | \psi_{i}\right\rangle=\lambda_{i}^{2}$. Consider for now only those $i$ for which $\lambda_{i} \neq 0$. For those $i$ define $\left|e_{i}\right\rangle \equiv\left|\psi_{i}\right\rangle / \lambda_{i}$, so the $\left|e_{i}\right\rangle$ are normalized. Moreover, they are orthogonal, since if $i \neq j$ then $\left\langle e_{i} | e_{j}\right\rangle=$ $\left\langle i\left|A^{\dagger} A\right| j\right\rangle / \lambda_{i} \lambda_{j}=\left\langle i\left|J^{2}\right| j\right\rangle / \lambda_{i} \lambda_{j}=0$

We have been considering $i$ such that $\lambda_{i} \neq 0$. Now use the Gram-Schmidt procedure to extend the orthonormal set $\left|e_{i}\right\rangle$ so it forms an orthonormal basis, which we also label $\left|e_{i}\right\rangle$. Define a unitary operator $U \equiv \sum_{i}\left|e_{i}\right\rangle\langle i|$. When $\lambda_{i} \neq 0$ we have $U J|i\rangle=\lambda_{i}\left|e_{i}\right\rangle=$ $\left|\psi_{i}\right\rangle=A|i\rangle$. When $\lambda_{i}=0$ we have $U J|i\rangle=0=\left|\psi_{i}\right\rangle$. We have proved that the action of $A$ and $U J$ agree on the basis $|i\rangle$, and thus that $A=U J$.

$J$ is unique, since multiplying $A=U J$ on the left by the adjoint equation $A^{\dagger}=J U^{\dagger}$ gives $J^{2}=A^{\dagger} A$, from which we see that $J=\sqrt{A^{\dagger} A}$, uniquely. A little thought shows that if $A$ is invertible, then so is $J$, so $U$ is uniquely determined by the equation $U=A J^{-1}$. The proof of the right polar decomposition follows, since $A=U J=U J U^{\dagger} U=K U$, where $K \equiv U J U^{\dagger}$ is a positive operator. Since $A A^{\dagger}=K U U^{\dagger} K=K^{2}$ we must have $K=\sqrt{A A^{\dagger}}$, as claimed.
\end{proof}

The singular value decomposition combines the polar decomposition and the spectral theorem.

\begin{corollary}
Corollary 2.4: (Singular value decomposition) Let $A$ be a square matrix. Then there exist unitary matrices $U$ and $V$, and a diagonal matrix $D$ with non-negative entries such that
$$
A=U D V
$$
The diagonal elements of $D$ are called the singular values of $A$.
\end{corollary}

\begin{proof}
    By the polar decomposition, $A=S J$, for unitary $S$, and positive $J$. By the spectral theorem, $J=T D T^{\dagger}$, for unitary $T$ and diagonal $D$ with non-negative entries. Setting $U \equiv S T$ and $V \equiv T^{\dagger}$ completes the proof.
\end{proof}

\begin{exercise}
Exercise 2.48: What is the polar decomposition of a positive matrix $P$? Of a unitary matrix $U$ ? Of a Hermitian matrix, $H$ ?
\end{exercise}

\begin{exercise}
Exercise 2.49: Express the polar decomposition of a normal matrix in the outer product representation.
\end{exercise}

\begin{exercise}
Exercise 2.50: Find the left and right polar decompositions of the matrix

$$
\left[\begin{array}{ll}
1 & 0 \\
1 & 1
\end{array}\right]
$$
\end{exercise}