Suppose $V, W$ are finite-dimensional vector spaces over a field, with dimensions $m$ and $n$, respectively. 
For any vector space $A$, let $L(A)$ denote the space of linear operators on $A$.

\begin{definition}[partial trace I]\label{defn:partial-trace-1}
The \textbf{partial trace over $W$} is an operator-valued function written as
$$
\operatorname{Tr}_W: \mathrm{L}(V \otimes W) \rightarrow \mathrm{L}(V),
$$
where $\otimes$ denotes the tensor product. It is defined as follows: Let $e_1, \ldots, e_m$, and $f_1, \ldots, f_n$, be bases for $V$ and $W$ respectively; then $T \in \mathrm{L}(V \otimes W)$ has a matrix representation
$$
a_{k \ell, i j}, \quad 1 \leq k, i \leq m, \quad 1 \leq \ell, j \leq n
$$
relative to the basis $e_k \otimes f_{\ell}$ of $V \otimes W$. Now fix indices $k, i$ in the range $1, \ldots, m$, consider the sum
$$
b_{k, i}=\sum_{j=1}^n a_{k j, i j}
$$
This gives a matrix $b_{k, i}$. The associated linear operator (of matrix $b_{k, i}$) on \textit{V} is independent of the choice of bases and is by definition the partial trace.
\end{definition}

Among physicists, this is often called \textbf{"tracing out" or "tracing over" $W$} to leave only an operator on $V$ in the context where $W$ and $V$ are Hilbert spaces associated with quantum systems (see below).