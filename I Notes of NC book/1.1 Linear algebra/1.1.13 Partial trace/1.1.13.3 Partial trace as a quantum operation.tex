The partial trace can be viewed as a quantum operation.

Consider a quantum mechanical system whose state space is the tensor product $H_A \otimes H_B$ of Hilbert spaces. A mixed state is described by a density matrix $\rho$, that is a non-negative trace-class operator of trace 1 on the tensor product $H_A \otimes H_B$. The partial trace of $\rho$ with respect to the system $B$, denoted by $\rho^A$, is called the reduced state of $\rho$ on system $A$. In symbols,
$$
\rho^A=\operatorname{Tr}_B \rho
$$

To show that this is indeed a sensible way to assign a state on the $A$ subsystem to $\rho$, we offer the following justification. 

Let $M$ be an observable on the subsystem $A$, then the corresponding observable on the composite system is $M \otimes I$. However one chooses to define a reduced state $\rho^A$, there should be consistency of measurement statistics. The expectation value of $M$ after the subsystem $A$ is prepared in $\rho^A$ and that of $M \otimes I$ when the composite system is prepared in $\rho$ should be the same, i.e. the following equality should hold:
$$
\operatorname{Tr}\left(M \cdot \rho^A\right)=\operatorname{Tr}(M \otimes I \cdot \rho) .
$$
We see that this is satisfied if $\rho^A$ is as defined above via the partial trace. Furthermore, such operation is unique.

\begin{proof}
    my proof is done on
\end{proof}