The tensor product is a way of putting vector spaces together to form larger vector spaces. This construction is crucial to understanding the quantum mechanics of multiparticle systems. The following discussion is a little abstract, and may be difficult to follow if you're not already familiar with the tensor product, so feel free to skip ahead now and revisit later when you come to the discussion of tensor products in quantum mechanics.

Suppose $V$ and $W$ are vector spaces of dimension $m$ and $n$ respectively. For convenience we also suppose that $V$ and $W$ are Hilbert spaces. Then $V \otimes W$ (read ' $V$ tensor

Box 2.2: The spectral decomposition - important!

The spectral decomposition is an extremely useful representation theorem for normal operators.

Theorem 2.1: (Spectral decomposition) Any normal operator $M$ on a vector space $V$ is diagonal with respect to some orthonormal basis for $V$. Conversely, any diagonalizable operator is normal.

Proof
The converse is a simple exercise, so we prove merely the forward implication, by induction on the dimension $d$ of $V$. The case $d=1$ is trivial. Let $\lambda$ be an eigenvalue of $M, P$ the projector onto the $\lambda$ eigenspace, and $Q$ the projector onto the orthogonal complement. Then $M=(P+Q) M(P+Q)=P M P+Q M P+$ $P M Q+Q M Q$. Obviously $P M P=\lambda P$. Furthermore, $Q M P=0$, as $M$ takes the subspace $P$ into itself. We claim that $P M Q=0$ also. To see this, let $|v\rangle$ be an element of the subspace $P$. Then $M M^{\dagger}|v\rangle=M^{\dagger} M|v\rangle=\lambda M^{\dagger}|v\rangle$. Thus, $M^{\dagger}|v\rangle$ has eigenvalue $\lambda$ and therefore is an element of the subspace $P$. It follows that $Q M^{\dagger} P=0$. Taking the adjoint of this equation gives $P M Q=0$. Thus $M=P M P+Q M Q$. Next, we prove that $Q M Q$ is normal. To see this, note that $Q M=Q M(P+Q)=Q M Q$, and $Q M^{\dagger}=Q M^{\dagger}(P+Q)=Q M^{\dagger} Q$. Therefore, by the normality of $M$, and the observation that $Q^{2}=Q$,

$$
\begin{aligned}
Q M Q Q M^{\dagger} Q & =Q M Q M^{\dagger} Q \\
& =Q M M^{\dagger} Q \\
& =Q M^{\dagger} M Q \\
& =Q M^{\dagger} Q M Q \\
& =Q M^{\dagger} Q Q M Q
\end{aligned}
$$

so $Q M Q$ is normal. By induction, $Q M Q$ is diagonal with respect to some orthonormal basis for the subspace $Q$, and $P M P$ is already diagonal with respect to some orthonormal basis for $P$. It follows that $M=P M P+Q M Q$ is diagonal with respect to some orthonormal basis for the total vector space.

In terms of the outer product representation, this means that $M$ can be written as $M=\sum_{i} \lambda_{i}|i\rangle\langle i|$, where $\lambda_{i}$ are the eigenvalues of $M,|i\rangle$ is an orthonormal basis for $V$, and each $|i\rangle$ an eigenvector of $M$ with eigenvalue $\lambda_{i}$. In terms of projectors, $M=\sum_{i} \lambda_{i} P_{i}$, where $\lambda_{i}$ are again the eigenvalues of $M$, and $P_{i}$ is the projector onto the $\lambda_{i}$ eigenspace of $M$. These projectors satisfy the completeness relation $\sum_{i} P_{i}=I$, and the orthonormality relation $P_{i} P_{j}=\delta_{i j} P_{i}$.

$W^{\prime}$ ) is an $m n$ dimensional vector space. The elements of $V \otimes W$ are linear combinations of 'tensor products' $|v\rangle \otimes|w\rangle$ of elements $|v\rangle$ of $V$ and $|w\rangle$ of $W$. In particular, if $|i\rangle$ and $|j\rangle$ are orthonormal bases for the spaces $V$ and $W$ then $|i\rangle \otimes|j\rangle$ is a basis for $V \otimes W$. We often use the abbreviated notations $|v\rangle|w\rangle,|v, w\rangle$ or even $|v w\rangle$ for the tensor product\\
$|v\rangle \otimes|w\rangle$. For example, if $V$ is a two-dimensional vector space with basis vectors $|0\rangle$ and $|1\rangle$ then $|0\rangle \otimes|0\rangle+|1\rangle \otimes|1\rangle$ is an element of $V \otimes V$.

By definition the tensor product satisfies the following basic properties:

(1) For an arbitrary scalar $z$ and elements $|v\rangle$ of $V$ and $|w\rangle$ of $W$,

$$
z(|v\rangle \otimes|w\rangle)=(z|v\rangle) \otimes|w\rangle=|v\rangle \otimes(z|w\rangle)
$$

(2) For arbitrary $\left|v_{1}\right\rangle$ and $\left|v_{2}\right\rangle$ in $V$ and $|w\rangle$ in $W$,

$$
\left(\left|v_{1}\right\rangle+\left|v_{2}\right\rangle\right) \otimes|w\rangle=\left|v_{1}\right\rangle \otimes|w\rangle+\left|v_{2}\right\rangle \otimes|w\rangle .
$$

(3) For arbitrary $|v\rangle$ in $V$ and $\left|w_{1}\right\rangle$ and $\left|w_{2}\right\rangle$ in $W$,

$$
|v\rangle \otimes\left(\left|w_{1}\right\rangle+\left|w_{2}\right\rangle\right)=|v\rangle \otimes\left|w_{1}\right\rangle+|v\rangle \otimes\left|w_{2}\right\rangle .
$$

What sorts of linear operators act on the space $V \otimes W$ ? Suppose $|v\rangle$ and $|w\rangle$ are vectors in $V$ and $W$, and $A$ and $B$ are linear operators on $V$ and $W$, respectively. Then we can define a linear operator $A \otimes B$ on $V \otimes W$ by the equation

$$
(A \otimes B)(|v\rangle \otimes|w\rangle) \equiv A|v\rangle \otimes B|w\rangle
$$

The definition of $A \otimes B$ is then extended to all elements of $V \otimes W$ in the natural way to ensure linearity of $A \otimes B$, that is,

$$
(A \otimes B)\left(\sum_{i} a_{i}\left|v_{i}\right\rangle \otimes\left|w_{i}\right\rangle\right) \equiv \sum_{i} a_{i} A\left|v_{i}\right\rangle \otimes B\left|w_{i}\right\rangle .
$$

It can be shown that $A \otimes B$ defined in this way is a well-defined linear operator on $V \otimes W$. This notion of the tensor product of two operators extends in the obvious way to the case where $A: V \rightarrow V^{\prime}$ and $B: W \rightarrow W^{\prime}$ map between different vector spaces. Indeed, an arbitrary linear operator $C$ mapping $V \otimes W$ to $V^{\prime} \otimes W^{\prime}$ can be represented as a linear combination of tensor products of operators mapping $V$ to $V^{\prime}$ and $W$ to $W^{\prime}$,

$$
C=\sum_{i} c_{i} A_{i} \otimes B_{i}
$$

where by definition

$$
\left(\sum_{i} c_{i} A_{i} \otimes B_{i}\right)|v\rangle \otimes|w\rangle \equiv \sum_{i} c_{i} A_{i}|v\rangle \otimes B_{i}|w\rangle .
$$

The inner products on the spaces $V$ and $W$ can be used to define a natural inner product on $V \otimes W$. Define

$$
\left(\sum_{i} a_{i}\left|v_{i}\right\rangle \otimes\left|w_{i}\right\rangle, \sum_{j} b_{j}\left|v_{j}^{\prime}\right\rangle \otimes\left|w_{j}^{\prime}\right\rangle\right) \equiv \sum_{i j} a_{i}^{*} b_{j}\left\langle v_{i} | v_{j}^{\prime}\right\rangle\left\langle w_{i} | w_{j}^{\prime}\right\rangle
$$

It can be shown that the function so defined is a well-defined inner product. From this inner product, the inner product space $V \otimes W$ inherits the other structure we are familiar with, such as notions of an adjoint, unitarity, normality, and Hermiticity.

All this discussion is rather abstract. It can be made much more concrete by moving\\
to a convenient matrix representation known as the Kronecker product. Suppose $A$ is an $m$ by $n$ matrix, and $B$ is a $p$ by $q$ matrix. Then we have the matrix representation:

$$
A \otimes B \equiv \overbrace{\left[\begin{array}{cccc}
A_{11} B & A_{12} B & \ldots & A_{1 n} B \\
A_{21} B & A_{22} B & \ldots & A_{2 n} B \\
\vdots & \vdots & \vdots & \vdots \\
A_{m 1} B & A_{m 2} B & \ldots & A_{m n} B
\end{array}\right]}^{n q} m m p
$$

In this representation terms like $A_{11} B$ denote $p$ by $q$ submatrices whose entries are proportional to $B$, with overall proportionality constant $A_{11}$. For example, the tensor product of the vectors $(1,2)$ and $(2,3)$ is the vector

$$
\left[\begin{array}{l}
1 \\
2
\end{array}\right] \otimes\left[\begin{array}{l}
2 \\
3
\end{array}\right]=\left[\begin{array}{l}
1 \times 2 \\
1 \times 3 \\
2 \times 2 \\
2 \times 3
\end{array}\right]=\left[\begin{array}{l}
2 \\
3 \\
4 \\
6
\end{array}\right]
$$

The tensor product of the Pauli matrices $X$ and $Y$ is

$$
X \otimes Y=\left[\begin{array}{cc}
0 \cdot Y & 1 \cdot Y \\
1 \cdot Y & 0 \cdot Y
\end{array}\right]=\left[\begin{array}{cccc}
0 & 0 & 0 & -i \\
0 & 0 & i & 0 \\
0 & -i & 0 & 0 \\
i & 0 & 0 & 0
\end{array}\right]
$$

Finally, we mention the useful notation $|\psi\rangle^{\otimes k}$, which means $|\psi\rangle$ tensored with itself $k$ times. For example $|\psi\rangle^{\otimes 2}=|\psi\rangle \otimes|\psi\rangle$. An analogous notation is also used for operators on tensor product spaces.

Exercise 2.26: Let $|\psi\rangle=(|0\rangle+|1\rangle) / \sqrt{2}$. Write out $|\psi\rangle^{\otimes 2}$ and $|\psi\rangle^{\otimes 3}$ explicitly, both in terms of tensor products like $|0\rangle|1\rangle$, and using the Kronecker product.

Exercise 2.27: Calculate the matrix representation of the tensor products of the Pauli operators (a) $X$ and $Z$; (b) $I$ and $X$; (c) $X$ and $I$. Is the tensor product commutative?

Exercise 2.28: Show that the transpose, complex conjugation, and adjoint operations distribute over the tensor product,

$$
(A \otimes B)^{*}=A^{*} \otimes B^{*} ;(A \otimes B)^{T}=A^{T} \otimes B^{T} ;(A \otimes B)^{\dagger}=A^{\dagger} \otimes B^{\dagger} .
$$

Exercise 2.29: Show that the tensor product of two unitary operators is unitary.

Exercise 2.30: Show that the tensor product of two Hermitian operators is Hermitian.

Exercise 2.31: Show that the tensor product of two positive operators is positive.

Exercise 2.32: Show that the tensor product of two projectors is a projector.

Exercise 2.33: The Hadamard operator on one qubit may be written as

$$
H=\frac{1}{\sqrt{2}}[(|0\rangle+|1\rangle)\langle 0|+(|0\rangle-|1\rangle)\langle 1|]
$$

Show explicitly that the Hadamard transform on $n$ qubits, $H^{\otimes n}$, may be written as

$$
H^{\otimes n}=\frac{1}{\sqrt{2^{n}}} \sum_{x, y}(-1)^{x \cdot y}|x\rangle\langle y|
$$

Write out an explicit matrix representation for $H^{\otimes 2}$.