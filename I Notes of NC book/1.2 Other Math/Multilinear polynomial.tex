
\paragraph{MULTIVARIATE POLYNOMIAL}

A multivariate polynomial is a polynomial in more than one variable, e.g., 
$$
P(x, y)
=a_{22} x^2 y^2+a_{21} x^2 y+a_{12} x y^2+a_{11} x y+a_{10} x+a_{01} y+a_{00}
$$
Note that there are monimials such as $a_{22} x^2 y^2, a_{21} x^2 y$. If we fix the variable $y$, then $x \mapsto P(x,y)$ is not a linear function.

\paragraph{MULTILINEAR POLYNOMIAL} %一定要注意区分!多重线性多项式函数的一般理解的(一元)多项式函数有很大的不同

\href{https://en.wikipedia.org/wiki/Multilinear_polynomial}{Multilinear polynomial - Wikipedia}

In algebra, a multilinear polynomial is a multivariate polynomial that is \textbf{linear (meaning affine) in each of its variables separately}, but not necessarily simultaneously. 

\textbf{It is a polynomial in which no variable occurs to a power of 2 or higher}; that is, each monomial is a constant times a product of distinct variables. 

For example 
$$
f(x, y, z)=3 x y+2.5 y -7 z
$$
is a multilinear polynomial of degree 2 (because of the monomial $3 x y$ ) whereas $$
f(x, y, z)=x^2+4 y
$$
is not. The \textit{degree} of a multilinear polynomial is the maximum number of distinct variables occurring in any monomial.

Multilinear polynomials can be understood as a multilinear map (specifically, a multilinear form) applied to the vectors $[1 \; \mathrm{x}],[1 \; \mathrm{y}]$, etc. The general form can be written as a tensor contraction:
\begin{align}
    f(x)&=\sum_{i_1=0}^1 \sum_{i_2=0}^1 \cdots \sum_{i_n=0}^1 a_{i_1 i_2 \cdots i_n} x_1^{i_1} x_2^{i_2} \cdots x_n^{i_n}\\
    &=\sum_{i=\left(i_1, \ldots ,i_n\right) \in\{0,1\}^n} a_i \prod_{k=1}^n x_k^{i_k}.
\end{align}
For example, in two variables:
$$
f(x, y)=\sum_{i=0}^1 \sum_{j=0}^1 a_{i j} x^i y^j=a_{00}+a_{10} x+a_{01} y+a_{11} x y=\left(\begin{array}{ll}
1 & x
\end{array}\right)\left(\begin{array}{cc}
a_{00} & a_{01} \\
a_{10} & a_{11}
\end{array}\right)\binom{1}{y}.
$$