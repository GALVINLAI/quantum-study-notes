The classical algorithm is to test distinct input values $\vec{x_1}, \vec{x_2}, \cdots$ until we find two input values $\vec{x_p}$ and $\vec{x_q}$ such that 

\begin{equation}
    f(\vec{x_p}) = f(\vec{x_q}),p \neq q.
\end{equation}

\begin{remark}
If $\vec{s} \neq \vec{0}$, $\vec{s}$ is determined once we find two distinct input values $\vec{x}$ and $\vec{y}$ such that $f(\vec{x})=f(\vec{y})$ and we can get $\vec{s} = \vec{x} \oplus \vec{y}$. 
\end{remark}
	\begin{proof}
		If we find two distinct $\vec{x}$ and $\vec{y}$ such that $f(\vec{x}) = f(\vec{y})$,  we get $\vec{y} =\vec{x} \oplus \vec{s}$. Then we have
		\begin{equation}
		\begin{aligned}
			\vec{y} & =\vec{x} \oplus \vec{s}, \\
			\vec{x} \oplus \vec{y} & =\vec{x} \oplus \vec{x} \oplus \vec{s}, \\
			\vec{x} \oplus \vec{y} & =\vec{0} \oplus \vec{s}(\because \vec{x} \oplus \vec{x}=\vec{0}), \\
			\vec{x} \oplus \vec{y} & =\vec{s}.
		\end{aligned}
\end{equation}
	\end{proof}

\begin{remark}
If $\vec{s} = \vec{0}$, we can't find two distinct input values $\vec{x}$ and $\vec{y}$ such that $f(\vec{x})=f(\vec{y})$ until we have tested $2^{n-1}+1$ input values.
\end{remark}

\begin{mdframed}
Given a special function $f(x)$ presented by Simon, suppose we have tested $m$ distinct input values $\vec{x_1}, \vec{x_2},\cdots, \vec{x_m}$ and we have

\begin{equation}
    f(\vec{x_i}) \neq f(\vec{x_j}), \forall i,j \in [m],i \neq j.
\end{equation} 

Then we can get

\begin{equation}
    \vec{s} \neq \vec{x_i} \oplus \vec{x_j}, \forall i,j \in [m],i \neq j.
\end{equation}

Therefore we have eliminated \textbf{at most} $\frac{m(m-1)}{2}$ possible values of $\vec{s}$ from $\{0,1\}^n$.
\end{mdframed}

\begin{remark}
Fewer possible values of $\vec{s}$ may have been eliminated if we test a new input value $\vec{w} = \vec{x} \oplus \vec{y} \oplus \vec{z}$, where $\vec{x}, \vec{y}, \vec{z}$ have already been tested.
\end{remark}

\begin{mdframed}
We don't know which binary string is $\vec{s}$ in $\{0,1\}^n$. When we give the computer two distinct input values $\vec{\alpha}$, $\vec{\beta}$ and get two distinct output values $f(\vec{\alpha}),f(\vec{\beta})$. Then we say $\vec{\alpha} \oplus \vec{\beta}$ has been eliminated from possible values of $\vec{s}$. At the beginning, the possible values of $\vec{s}$ can be all nonzero elements in $\{0,1\}^n$.
\end{mdframed}

\begin{proof}
Given a new input value $\vec{w}$ such that $\vec{w}=\vec{x} \oplus \vec{y} \oplus \vec{z}$, we have 
\begin{equation}
\begin{aligned}
	& \vec{w} \oplus \vec{z}=\vec{x} \oplus \vec{y} \oplus \vec{z} \oplus \vec{z}, \\
	& \vec{w} \oplus \vec{z}=\vec{x} \oplus \vec{y} \oplus \vec{0} \quad(\because \vec{z} \oplus \vec{z}=0), \\
	& \vec{w} \oplus \vec{z}=\vec{x} \oplus \vec{y}.
\end{aligned}
\end{equation}
Therefore, $\vec{w} \oplus \vec{z}$ has already been eliminated from $\{0,1\}^n$.
\end{proof}

\begin{example}
Consider the binary function $f_3(x)$
\begin{equation}
    \begin{array}{|r|r|}
	\hline x & f_3(x) \\
	\hline 0000,1001 & 1111 \\
	\hline 0001,1000 & 0001 \\
	\hline 0010,1011 & 1110 \\
	\hline 0011,1010 & 1101 \\
	\hline 0100,1101 & 0000 \\
	\hline 0101,1100 & 0101 \\
	\hline 0110,1111 & 1010 \\
	\hline 0111,1110 & 1001 \\
	\hline
\end{array}
\end{equation}
where $s=1001$. If we have tested $0000$, $0001$ and $0010$, we get

\begin{equation}
\begin{aligned}
	\vec{s} \neq 0000 \oplus 0001 = 0001, \\
	\vec{s} \neq 0000 \oplus 0010 = 0010, \\
	\vec{s} \neq 0001 \oplus 0010 = 0011. \\
\end{aligned}
\end{equation}
Then we test the new input value $0011$, which satisfies

\begin{equation}
\begin{aligned}
0011 &= 0000 \oplus 0001 \oplus 0010,\\
0011 \oplus 0010 &= 0000 \oplus 0001 \oplus0010 \oplus0010 ,\\
0011 \oplus 0010 &= 0000 \oplus 0001 (\because 0010 \oplus 0010 = 0000).
\end{aligned}
\end{equation}
Since input values $0011$ and $0010$ have distinct output values, we get 

\begin{equation}
    \vec{s} \neq 0011 \oplus 0010 = 0000 \oplus 0001 = 0001.
\end{equation}
Here we note that $0001$ has already been verified that it can't be $\vec{s}$ by $0000$ and $0001$. After testing the forth binary string $0011$, we have
\begin{equation}
\begin{aligned}
	\vec{s} \neq 0011 \oplus 0000 = 0011, \\
	\vec{s} \neq 0011 \oplus 0001 = 0010, \\
	\vec{s} \neq 0011 \oplus 0010 = 0001. \\
\end{aligned}
\end{equation}
So, in fact we've only eliminated $3 < \frac{4 \times 3}{2}$ possible values for $\vec{s}$ from $\{0,1\}^n$.
\end{example}