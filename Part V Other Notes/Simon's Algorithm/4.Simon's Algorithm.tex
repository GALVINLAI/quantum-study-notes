\begin{figure}
    \centering
    \begin{quantikz}[slice all,slice style=red,slice label
        style={inner sep=1pt,anchor=south west,rotate=0}]
    &	\lstick{$\ket{0}^{\otimes n}$}  & \gate{H^{\otimes n}} & \gate[2]{U_f}  &\gate{H^{\otimes n}} &\meter{} & \qw  \\
    &	\lstick{$\ket{0}^{\otimes n}$}  & \qw                  & 
    \qw      	   &   \qw               & \qw     & \qw  
    \end{quantikz}
    \caption{The Circuit of Simon's Algorithm}
    \label{fig6}
\end{figure}

From an algebraic point of view, the circuit is described by the equation
\begin{equation}
    \left(H^{\otimes n} \otimes I^{\otimes n}\right) U_f\left(H^{\otimes n} \otimes I^{\otimes n}\right)\left(|\vec{0}\rangle_n \otimes|\vec{0}\rangle_n\right).
\end{equation}
Next is the decomposition of the process. First we get 

\begin{equation}
\begin{aligned}
     \left|\psi_1\right\rangle 
    &=\left|\vec{0}\right\rangle_n \left|\vec{0}\right\rangle_n \\
     \left|\psi_2\right\rangle
    &=\left(H^{\otimes n} \otimes I^{\otimes n}\right)\left(|\vec{0}\rangle_n \otimes|\vec{0}\rangle_n\right)\\
    &=\left(H^{\otimes n} \left|\vec{0}\right\rangle_n \right) \otimes |\vec{0}\rangle_n\\
    &=\underbrace{\left(\frac{1}{\sqrt{2}}|0\rangle+\frac{1}{\sqrt{2}}|1\rangle\right) \otimes \cdots \otimes\left(\frac{1}{\sqrt{2}}|0\rangle+\frac{1}{\sqrt{2}}|1\rangle\right)}_n \otimes |\vec{0}\rangle_n\\
    &=\frac{1}{\sqrt{2^n}} \sum_{\vec{x} \in\{0,1\}^n}|\vec{x}\rangle_n|\vec{0}\rangle_n.\\
\end{aligned}
\end{equation}
Before continuing, we need to understand the black box $U_f$.

\begin{figure}
    \centering
    \begin{quantikz}
        \lstick{$|\vec{x}\rangle_n$} & \qw & \gate[2][1cm]{U_f} & \qw & \qw & \rstick{$|\vec{x}\rangle_n$} \\
        \lstick{$|\vec{y}\rangle_n$} & \qw &                    & \qw & \qw & \rstick{$|\vec{y}\oplus f(\vec{x})\rangle_n$} \\
    \end{quantikz}
    \caption{The Black Box $U_f$}
    \label{fig7}
\end{figure}
\textbf{Figure}\ref{fig7} is a black box (oracle), where $\vec{x}, \vec{y}\in \{0,1\}^n$. Black box $U_f$ takes $|\vec{x}\rangle_n \otimes |\vec{y}\rangle_n$ as input, then outputs $|\vec{x}\rangle_n \otimes|\vec{y}\oplus f(\vec{x})\rangle_n$. If $\vec{y} = \vec{0}$, the output is $|\vec{x}\rangle_n \otimes |f(\vec{x})\rangle_n$ ($\because \vec{0} \oplus f(\vec{x}) = f(\vec{x})$). Then we get

\begin{equation}
    \left|\psi_3\right\rangle=\frac{1}{\sqrt{2^n}} \sum_{\vec{x} \in\{0,1\}^n}|\vec{x}\rangle_n|f(\vec{x})\rangle_n.
\end{equation}
After $|\psi_3\rangle$, we measure the second register. Assuming the measurement of second register is $|f(\vec{z})\rangle$, then the quantum state $|\psi_3\rangle$ collapses to
\begin{equation}
    |\psi_4\rangle = \left(\frac{|\vec{z}\rangle+|\vec{z} \oplus \vec{s}\rangle}{\sqrt{2}}\right) \otimes|f(\vec{z})\rangle.
\end{equation}
Then he apply $H^{\otimes n} \otimes I^{\otimes n}$ to the $|\psi_4\rangle$. 

\begin{remark}[mention again]
Given an n-qubit quantum state $|\vec{x}\rangle_n$, where $\vec{x} \in \{0,1\}^n$, applying n Hadamard gates to the $|x\rangle_n$ yields 
\begin{equation}
    H^{\otimes n}|\vec{x}\rangle=\sum_{\vec{y} \in\{0,1\}^n}(-1)^{\vec{y} \bullet \vec{x}}|\vec{y}\rangle.
\end{equation}

Here we suppose $\vec{x} = x_1 x_2 \cdots x_n$ and $\vec{y} = y_1 y_2 \cdots y_n$, where $\forall i \in [n],x_i,y_i \in \{0,1\}$. The $\bullet$ is defined as 
\begin{equation}
    \vec{y} \bullet \vec{x} = y_1 x_1 + y_2 x_2 + \cdots + y_n x_n.
\end{equation}
\end{remark}

\begin{proof}
Applying an Hardmard gate to a 1-qubit state $|\lambda\rangle$, where $\lambda = \{0,1\}$, we get 
\begin{equation}
    H|\lambda\rangle = \frac{|0\rangle + (-1)^{\lambda}|1\rangle}{\sqrt{2}}.
\end{equation}
Applying n Hardmard gates to an n-qubit state $|\vec{x}\rangle_n$, where $\vec{x} = \{0,1\}^n$, we get 

\begin{equation}
\begin{aligned}
H ^{\otimes n}|\vec{x}\rangle_n
& = \left(H\left|x_1\right\rangle\right) \otimes\left(H\left|x_2\right\rangle\right) \otimes \cdots \otimes\left(H\left|x_n\right\rangle\right) \\
& = \left(\frac{|0\rangle+(-1)^{x_1}|1\rangle}{\sqrt{2}}\right) \otimes\left(\frac{|0\rangle+(-1)^{x_2}|1\rangle}{\sqrt{2}}\right) \otimes \cdots \otimes\left(\frac{|0\rangle+(-1)^{x_n}|1\rangle}{\sqrt{2}}\right) \\
& = \frac{1}{\sqrt{2^n}}\left((-1)^{0}|0 \cdots 00\rangle_n+(-1)^{x_n}|0 \cdots 01\rangle_n+\cdots+(-1)^{\left(x_1+\cdots+x_n\right)}|1 \cdots 11\rangle_n\right) \\
& = \frac{1}{\sqrt{2^n}}\left((-1)^{\vec{0} \bullet \vec{x}}|\vec{0}\rangle_n+(-1)^{(0 \cdots 01) \bullet \vec{x}}|0 \cdots 01\rangle_n+\cdots+(-1)^{(1\cdots \cdot 11) \bullet \vec{x}}|1\cdots 11\rangle_n\right) \\
& = \frac{1}{\sqrt{2^n}}\sum_{\vec{y} \in\{0,1\}^n}(-1)^{\vec{y} \bullet \vec{x}}|\vec{y}\rangle.
\end{aligned}
\end{equation}
\end{proof}
According to the property of Hadamard Gate, if we apply $n$ Hadamard gates on a $n$-qubit quantum state $|\vec{x}\rangle_n$, where $\vec{x} \in \{0,1\}^n$, we will get

\begin{equation}
    H^{\otimes n}|\vec{x}\rangle_n = \frac{1}{\sqrt{2^n}} \sum_{\vec{y} \in\{0,1\}^n}(-1)^{\vec{y} \bullet \vec{x}}|\vec{y}\rangle_n.
\end{equation} 
Apply $H^{\otimes n} \otimes I^{\otimes n}$ to the state $|\psi_3\rangle$ and we get 

\begin{equation}
\begin{aligned}
|\psi_4\rangle 
& =\left(H^{\otimes n} \otimes I^{\otimes n}\right) \left(\frac{1}{\sqrt{2^n}} \sum_{\vec{x} \in\{0,1\}^n}|\vec{x}\rangle_n \otimes |f(\vec{x})\rangle_n\right) \\
& =\frac{1}{2^n} \sum_{\vec{x} \in \{0, 1\}^n}\left(\sum_{\vec{y} \in \{0, 1\}^n}(-1)^{\vec{y} \bullet \vec{x}}|\vec{y}\rangle_n \otimes|f(\vec{x})\rangle_n\right)\\
& =\frac{1}{2^n} \sum_{\vec{y} \in \{0, 1\}^n}\left(\sum_{\vec{x} \in \{0, 1\}^n}(-1)^{\vec{y} \bullet \vec{x}}|\vec{y}\rangle_n \otimes|f(\vec{x})\rangle_n\right).
\end{aligned}
\end{equation}
Here we consider the situation $\vec{s} \neq \vec{0}$ and we have
\begin{equation}
    |\vec{y}\rangle_n \otimes|f(\vec{x})\rangle_n=|\vec{y}\rangle_n \otimes|f(\vec{x} \oplus \vec{s})\rangle_n.
\end{equation}
In this case the binary function of Simon's problem is two-to-one. Let $R$ be a set with the following property:$\forall \vec{x} \in\{0,1\}^n, R$ contains either $\vec{x}$ or $\vec{x} \oplus \vec{s}$, but \textbf{not both}. Then we rewrite $|\psi_4\rangle_n$ as

\begin{equation}
    |\psi_4\rangle_n = \frac{1}{2^{n}} \sum_{\vec{y} \in \{0,1\}^n}\left(\sum_{\vec{x} \in R} \left((-1)^{\vec{y} \bullet \vec{x}} +(-1)^{\vec{y} \bullet(\vec{x} \oplus \vec{s})}\right)|\vec{y}\rangle_n \otimes |f(\vec{x})\rangle_n\right).
\end{equation}
\begin{remark}\label{binary string 1}
Given there binary strings $\vec{y}, \vec{x}, \vec{s} \in \{0, 1\}^n$, we have

\begin{equation}
    (-1)^{[\vec{y} \bullet (\vec{x} \oplus \vec{s})]} 
= (-1)^{(\vec{y} \bullet \vec{x}+\vec{y} \bullet \vec{s})}.
\end{equation}
\end{remark}

\begin{proof}
By observing $\vec{y} \bullet(\vec{x} \oplus \vec{s})$, we have the following result. Here $y_i$, $x_i$ and $s_i$ are $i$-th elements of $\vec{y}$, $\vec{x}$ and $\vec{s}$ respectively.

\begin{equation}
    \begin{array}{|c|c|c|c|c|}
\hline y_i & x_i & s_i & y_i\left(x_i \oplus s_i\right) & y_i x_i+y_i s_i\\
\hline 0 & 0 & 0 & 0 & 0 \\
\hline 0 & 0 & 1 & 0 & 0 \\
\hline 0 & 1 & 0 & 0 & 0 \\
\hline 0 & 1 & 1 & 0 & 0 \\
\hline 1 & 0 & 0 & 0 & 0 \\
\hline 1 & 0 & 1 & 1 & 1 \\
\hline 1 & 1 & 0 & 1 & 1 \\
\hline 1 & 1 & 1 & 0 & \textbf{2} \\
\hline
\end{array},
\end{equation}
So it is obvious that

\begin{equation}
    y_i\left(x_i \oplus s_i\right) = y_ix_i + y_is_i \pmod{2}.
\end{equation}
Then we get

\begin{equation}
\begin{aligned}
&  \vec{y} \bullet (\vec{x} \oplus \vec{s}) \pmod{2}\\
= & \sum_{i=1}^n y_i \left(x_i \oplus s_i\right) \pmod{2} \\
= & \sum_{i=1}^n\left(y_i x_i+y_i s_i\right) \pmod{2} \\
= & \vec{y} \bullet \vec{x}+\vec{y} \bullet \vec{s} \pmod{2}.
\end{aligned}
\end{equation}
Since the positive or negative of $(-1)^a$ depends only on the parity of $a$, we have

\begin{equation}
\begin{aligned}
& (-1)^{[\vec{y} \bullet (\vec{x} \oplus \vec{s})]} \\
= & (-1)^{[\vec{y} \bullet (\vec{x} \oplus \vec{s}) \pmod{2}]} \\
= & (-1)^{[\vec{y} \bullet \vec{x}+\vec{y} \bullet \vec{s} \pmod{2}]} \\
= & (-1)^{(\vec{y} \bullet \vec{x}+\vec{y} \bullet \vec{s})}. \\
\end{aligned}
\end{equation}
\end{proof}
Then we rewrite $|\psi_4\rangle_n$ as

\begin{equation}
\begin{aligned}
|\psi_4\rangle_n 
& =\frac{1}{2^{n}} \sum_{\vec{y} \in \{0,1\}^n}\left(\sum_{\vec{x} \in R} \left((-1)^{\vec{y} \bullet \vec{x}} +(-1)^{\vec{y} \bullet(\vec{x} \oplus \vec{s})}\right)|\vec{y}\rangle_n \otimes |f(\vec{x})\rangle_n\right) \\
& = \frac{1}{2^n} \sum_{\vec{y} \in \{0,1\}^n} \left( \sum_{\vec{x} \in R} \left((-1)^{\vec{y} \bullet \vec{x}} + (-1)^{\vec{y} \bullet \vec{x}} (-1)^{\vec{y} \bullet \vec{s}}\right) |\vec{y}\rangle_n \otimes |f(\vec{x})\rangle_n \right)(\because \mathbf{remark} \ref{binary string 1})\\
& = \frac{1}{2^n} \sum_{\vec{y} \in \{0,1\}^n}\left(\sum_{\vec{x} \in R} \left((-1)^{\vec{y} \bullet \vec{x}}\left(1+(-1)^{\vec{y} \bullet \vec{s}}\right)\right)|\vec{y}\rangle_n \otimes|f(\vec{x})\rangle_n\right).
\end{aligned}
\end{equation}
Therefore the probability of obtaining $\vec{y}$ after measuring the first $n$ qubits is

\begin{equation}
    \sum_{\vec{x} \in R}\left(\frac{(-1)^{\vec{y} \bullet \vec{x}}\left(1+(-1)^{\vec{y} \bullet \vec{s}}\right)}{2^n}\right)^2=2^{n-1}\left(\frac{1+(-1)^{\vec{y} \bullet \vec{s}}}{2^n}\right)^2=
\begin{cases}
\frac{1}{2^{n-1}} & \text{if } \vec{y} \bullet \vec{s} = 0 \pmod{2} \\
0 & \text{if } \vec{y} \bullet \vec{s} = 1 \pmod{2},
\end{cases}
\end{equation}
where the multiplication factor $2^{n-1}$ comes from the fact that $|R|=\frac{2^n}{2}$. The probability of the observed outcome of $|\psi_4\rangle$ shows that we can only get $\vec{y}$ that satisfy $\vec{y} \bullet \vec{s} = 0 \pmod{2}$. If we have measured $n-1$ different $\vec{y}^{(1)}, \vec{y}^{(2)},..., \vec{y}^{(n-1)}$, we can find $\vec{s}$ by solving the linear equation 

\begin{equation}
    \begin{cases}
y_1^{(1)} s_1+y_2^{(1)} s_2+\cdots+y_n^{(1)} s_n=0 & \pmod{2}, \\ 
y_1^{(2)} s_1+y_2^{(2)} s_2+\cdots+y_n^{(2)} s_n=0 & \pmod{2}, \\ 
\vdots \\ 
y_1^{(n-1)} s_1+y_2^{(2)} s_2+\cdots+y_n^{(2)} s_n=0 & \pmod{2}, \\
\vec{s} \neq \vec{0}, & 
\end{cases}
\end{equation}
where $y_i^{(j)}$ represents the $i$-th element of the $\vec{y}^{(j)}$, $s_k$ represents the $k$-th element of the $\vec{s}$, $\vec{s} \neq \vec{0}$. After solving this linear equation, we get $s_1, s_2, \cdots, s_n$. The problem is solved. A half of $\vec{y} \in \{0,1\}^n$ satisfy $\vec{y} \bullet \vec{s}=1$. The other half of $\vec{y} \in \{0,1\}^n$ satisfy $\vec{y} \bullet \vec{s}=0$. We denote 
\begin{equation}
\begin{aligned}
        & A=\left\{\vec{y} \mid \vec{y} \bullet \vec{s} = 1, \vec{y} \in\{0,1\}^n\right\}, \\
        & B=\left\{\vec{y} \mid \vec{y} \bullet \vec{s} = 0, \vec{y} \in\{0,1\}^n\right\}.
    \end{aligned}
\end{equation}
According to the probability of obtaining $\vec{k}$, we conclude that we can only obtain $\vec{k}$ which is in the set $B$. Assuming the measurement of the first n qubits is $|\vec{k}\rangle_n$, the $|\psi_4\rangle_n$ collapses to

\begin{equation}
\begin{aligned}
|\psi_5\rangle_n 
& = \frac{1}{\sqrt{2^n}} \sum_{\vec{x} \in\{0,1\}^n}(-1)^{\vec{k} \bullet \vec{x}}|\vec{k}\rangle_n \otimes|f(\vec{x})\rangle_n\\
& = |\vec{k}\rangle_n \otimes\left(\frac{1}{\sqrt{2^n}} \sum_{\vec{x} \in\{0,1\}^n}(-1)^{\vec{k} \bullet \vec{x}}|f(\vec{x})\rangle_n\right).
\end{aligned}
\end{equation}
After measuring the first register, all of the $\vec{y}$ we can get are in set $B$. 

% 	\subsection{A Example of Simon's Algorithm}
% 	Consider we have a binary function $f(x)$ such as
% 	\begin{equation}
    % 	\begin{array}{|r|r|}
% 		\hline x & f(x) \\
% 		\hline 0000,1001 & 1111 \\
% 		\hline 0001,1000 & 0001 \\
% 		\hline 0010,1011 & 1110 \\
% 		\hline 0011,1010 & 1101 \\
% 		\hline 0100,1101 & 0000 \\
% 		\hline 0101,1100 & 0101 \\
% 		\hline 0110,1111 & 1010 \\
% 		\hline 0111,1110 & 1001 \\
% 		\hline
% 	\end{array}
%
\end{equation}
% 	, where $s=1001$. Its quantum circuit diagram is shown in \textbf{Figure}\ref{fig8}.
% 	\begin{figure}
% 		\centering
% 		\begin{quantikz}
% 			\lstick{$\ket{0}^{\otimes 4}$} & \gate{H^{\otimes 4}} & \gate[2]{U_f}&\gate{H^{\otimes 4}} &\meter{} &  \\
% 			\lstick{$\ket{0}^{\otimes 4}$} & \qw                & 
% 			\qw      &            \qw & \qw \\
% 		\end{quantikz}
% 		\caption{}
% 		\label{fig8}
% 	\end{figure}
% 	And its quantum state analysis is as follows.
% 	\begin{equation}
% 	\begin{aligned}
% 		\left|\psi_1\right\rangle 
% 		& =\left|0\right\rangle_4\left|0\right\rangle_4=|0000\rangle|0000\rangle, \\
% 		\left|\psi_2\right\rangle 
% 		& =H^{\otimes 4}|0000\rangle|0000\rangle=\frac{1}{4}\sum_{x \in\{0,1\}^4}|x\rangle|0000\rangle \\
% 		& =\frac{|0000\rangle+|0001\rangle+|0010\rangle+\cdots+| 1111\rangle}{4}|0000\rangle,\\
% 		\left|\psi_3\right\rangle 
% 		& =\frac{1}{4} \sum_{x \in\{0,1\}^4}|x\rangle|f(x)\rangle, \\
% 		\left|\psi_4\right\rangle
% 		& =  \frac{1}{16} \sum_{\vec{y} \in \{0,1\}^4}\left(\sum_{\vec{x} \in R} \left((-1)^{\vec{y} \bullet \vec{x}}\left(1+(-1)^{\vec{y} \bullet \vec{s}}\right)\right)|\vec{y}\rangle_4 \otimes|f(\vec{x})\rangle_4\right)
% 	\end{aligned}
%
\end{equation}
% , where $R = \{0000,0001,0010,0011,0100,0101,0110,0111\}$.
% 	
% 	Here we assume that the measurement of the first four qubits is $1010$. According to the function value table, its corresponding $\vec{x}$ and $\vec{x} \oplus \vec{s}$ are $0110$ and $1111$. Then we can get
% 	\begin{equation}
% 	\begin{aligned}
% 		\left|\psi_4\right\rangle
% 		&=\frac{\mid 0110)+|1111\rangle}{\sqrt{2}}|1010\rangle, \\
% 		\left|\psi_5\right\rangle
% 		&=H^{\otimes 4}\left[\frac{|01 10\rangle+|1111\rangle}{\sqrt{2}}\right]\\
% 		&=\frac{|0000\rangle-|0010\rangle-|0100\rangle+|0110\rangle+|1001\rangle-|1011\rangle-|1101\rangle+|1111\rangle}{\sqrt{2^3}}.
% 	\end{aligned}
%
\end{equation}
% 	After measurement for the first register, assume we get three different $y$, $y^{(1)}=0010$, $y^{(2)}=0100$, and $y^{(3)}=1001$. Then we can find $s$ by solving the linear equation 
% 	\begin{equation}
    % 	\left[\begin{array}{llll}
% 		0 & 0 & 1 & 0 \\
% 		0 & 1 & 0 & 0 \\
% 		1 & 0 & 0 & 1 \\
% 		0 & 0 & 0 & 0
% 	\end{array}\right]\left[\begin{array}{l}
% 		s_1 \\
% 		s_2 \\
% 		s_3 \\
% 		s_4
% 	\end{array}\right]=\left[\begin{array}{l}
% 		0 \\
% 		0 \\
% 		0 \\
% 		0
% 	\end{array}\right].
%
\end{equation}
% 	After solving the equation, we get $s_1=1$, $s_2=0$, $s_3=0$, $s_4=1$, so $s$ is $1001$ because of $s\neq 0$. Finally we simplify this problem from the time complex $\mathcal{O}(2^{n-1})$ to the time complex $\mathcal{O}(n)$.

% \appendix

% \medskip
% \bibliography{ref.bib}{} % Entries are in the ref.bib file
% \bibliographystyle{plain} % We choose the "plain" reference style

