  We recall that the NES is defined as

\begin{equation}
    M \Delta y = \sigma.
\end{equation}
Then we multiply both left and right sides of the NES by $R=D_{\mathcal{B}}^{-1} A_{\mathcal{B}}^{-1}$.
%where $D_{\mathcal{B}} =\left(X_{\mathcal{B}}\right)^{1 / 2}\left(S_{\mathcal{B}}\right)^{-1 / 2}$.
After multiplying, the NES becomes

\begin{equation}
    \label{NES times R in the left side}
    R M R^{\top} R^{-\top} \Delta y=R\sigma.
\end{equation}
Then we naturally get the definition of the modified normal equation system. 
\begin{definition}
Since $R\in \mathbb{R}^{m \times m}$ is an invertible matrix, problem (\ref{NES times R in the left side}) and the NES are equivalent. Further simplification yields the Modified Normal Equation System (MNES)
\begin{equation}
    \label{MNES}
    \hat{M} z=\hat{\sigma},
\end{equation}
where
\begin{equation}
\begin{aligned}
    & \hat{M} = R M R^{\top},\\
    & z = R^{-\top}\Delta y,\\
    & \hat{\sigma} = R \sigma.
\end{aligned}
\end{equation}
\end{definition}