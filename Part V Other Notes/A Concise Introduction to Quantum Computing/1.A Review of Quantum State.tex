In this handout, \textit{tenor product} and \textit{Kronecker product} refer to the same concept. The following phrases have the same meaning.
\begin{itemize}
    \item state of a $q$-qubit quantum register
    \item state of a $q$-qubit register
    \item state of a $q$-qubit
    \item $q$-qubit state
    \item $q$-qubit
    \item $q$ qubits
    \item a \textbf{quantum state} $|\psi\rangle \in\left(\mathbb{C}^2\right)^{\otimes q}$
    \item (Assumption 2.1 in original paper) a \textbf{unit vector} $|\psi\rangle$ in the tensor space $\left(\mathbb{C}^2\right)^{\otimes q}=\underbrace{\mathbb{C}^2 \otimes \cdots \otimes \mathbb{C}^2}_{q \text { times }}$, or we can write
    \begin{equation}
    |\psi\rangle=\sum_{\vec{\jmath} \in\{0,1\}^q} \alpha_{\vec{\jmath}}|\vec{\jmath}\rangle_q \text { with } \alpha_{\vec{\jmath}} \in \mathbb{C} \text { and } \sum_{\vec{\jmath} \in\{0,1\}^q}\left|\alpha_{\vec{\jmath}}\right|^2=1.
\end{equation}

\end{itemize}

\begin{remark}
    Mathematically, $q$-qubit is \textit{just} the unit vector in the complex Euclidean space $\left(\mathbb{C}^2\right)^{\otimes q} \equiv \mathbb{C}^{2^q}$. Note that it potentially has a unit length constraint. The usual linear operations, e.g., $|\psi\rangle + |\phi\rangle$ , will break this property, but tensor product will not. See Lemma \ref{lem1}.
\end{remark}

\begin{remark}
    Here a manifold appears. The set of all 1-qubit is a \textbf{complex sphere}, $\left\{|\psi\rangle \in \mathbb{C}^{2}:\||\psi\rangle\|=1\right\}$.
\end{remark}

\begin{remark}
    Don't be confused: ''a $q$-qubits state'' and ''q 1-qubit states'' are different concepts.
\end{remark}

The next lemma shows that the tensor product of two unit vectors is also a unit vector.
\begin{lemma}\label{lem1}
 Let $\mathbf{x} \in \mathbb{C}^m,\mathbf{y} \in \mathbb{C}^n$ and $\sum_{i=1}^m\left|x_i\right|^2=1, \sum_{j=1}^n\left|y_j\right|^2=1.$ 
 If we take
 $\mathbf{z} \triangleq \mathbf{x} \otimes \mathbf{y}\in\mathbb{C}^{mn}$, then it implies $\sum_{k=1}^{mn}\left|z_k\right|^2=1$. (See proof below.)
\end{lemma}

Thus we get a trivial result as below. We say that the tensor product of two quantum states is also a quantum state. 

\begin{corollary}\label{cor1}
If both $|\psi\rangle \in \left(\mathbb{C}^2\right)^{\otimes q_{1}}$ and $|\phi\rangle  \in \left(\mathbb{C}^2\right)^{\otimes q_{2}}$ are in a quantum state, $|\psi\rangle \otimes|\phi\rangle \in \left(\mathbb{C}^2\right)^{\otimes (q_{1}+q_{2})}$ is in a quantum state as well.
\end{corollary}

\begin{proof}[Proof of lemma \ref{lem1}]
    Let $\mathbf{x} \in \mathbb{C}^m,\mathbf{y} \in \mathbb{C}^n$ and $\sum_{i=1}^m\left|x_i\right|^2=1, \sum_{j=1}^n\left|y_j\right|^2=1.$
    By definition of tensor product, one has
    \begin{equation}
    \mathbf{z} \triangleq \mathbf{x} \otimes \mathbf{y}=\left[\begin{array}{c}
    x_1 \mathbf{y} \\
    x_2 \mathbf{y} \\
    \vdots \\
    x_m \mathbf{y}
    \end{array}\right], \text { where } \mathbf{y}=\left[\begin{array}{c}
    y_1 \\
    y_2 \\
    \vdots \\
    y_n
    \end{array}\right].
\end{equation}
    By observing the components of the vector $\mathbf{z}$, we can obtain that
    \begin{equation}
        \begin{aligned}
        & \sum_{k=1}^{m n}\left|z_k\right|^2=\sum_{j=1}^n\left|x_1 y_j\right|^2+\sum_{j=1}^n\left|x_2 y_j\right|^2+\cdots+\sum_{j=1}^n\left|x_m y_j\right|^2 \\
        & =\left(\sum_{j=1}^n\left|y_j\right|^2\right)\left(\left|x_1\right|^2+\left|x_2\right|^2+\cdots+\left|x_m\right|^2\right) \\
        & =\left(\sum_{j=1}^n\left|y_j\right|^2\right)
        \left(\sum_{i=1}^m\left|x_i\right|^2\right) \\
        & =1.
        \end{aligned}
\end{equation}
    The proof is complete.
\end{proof}

%Note that $\left(\mathbb{C}^2\right)^{\otimes q}$ is a $2^q$-dimensional space because it is a tensor product of $\mathbb{C}^2$.

