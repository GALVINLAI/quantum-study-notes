\subsubsection{Applying an Operation on a 1-qubit State is a Linear Transform}

The input state is the 1-qubit state $|\psi\rangle=\alpha|0\rangle+\beta|1\rangle$, where $|\alpha|^2+|\beta|^2=1$. Then the output state is $U|\psi\rangle$, which means that apply the unitary matrix $U\in \mathbb{C}^{2\times2}$ to the quantum 1-qubit state $|\psi\rangle\in\mathbb{C}^2$.

\vspace{0.5cm}

\begin{center}
    \begin{quantikz}
        \lstick{\ket{\psi}} & \gate{U} & \qw \rstick{$U\ket{\psi}$}\\
    \end{quantikz}
\end{center}

\vspace{0.5cm}

Write 1-qubit state $\alpha|0\rangle+\beta|1\rangle$ in a vector notation as
\begin{equation}
    \alpha\left[\begin{array}{l}
1 \\
0
\end{array}\right]+\beta\left[\begin{array}{l}
0 \\
1
\end{array}\right]=
\left[\begin{array}{l}
\alpha \\
\beta
\end{array}\right].
\end{equation}

Then the we can regard the operation on the 1-qubit state as

\begin{equation}
    U|\psi\rangle=
\left[\begin{array}{ll}
u_{11} & u_{12} \\
u_{21} & u_{22}
\end{array}\right]\left[\begin{array}{l}
\alpha \\
\beta
\end{array}\right]=\left[\begin{array}{c}
\alpha u_{11}+\beta u_{12} \\
\alpha u_{21}+\beta u_{22}
\end{array}\right]=\left(\alpha u_{11}+\beta u_{12}\right)|0\rangle+\left(\alpha u_{21}+\beta u_{22}\right)|1\rangle.
\end{equation}

\vspace{0.5cm}

\subsubsection{The Writing Order of in the Mathematical Expression}

  Circuit diagrams are read from left to right, but the matrices corresponding to the gates should be written from right to left in the mathematical expression describing the circuit. For example, the outcome of the circuit is the 1-qubit state $C B A|\psi\rangle$, the reading order is $A$, $B$, and $C$.
  \vspace{0.5cm}

\begin{center}
    \begin{quantikz}
        \lstick{\ket{\psi}}& \gate{A}&\gate{B}&\gate{C}
    & \rstick{$CBA$\ket{\psi}}\qw
    \end{quantikz}
\end{center}
\vspace{0.5cm}

\subsubsection{Quantum Circuit Lines without Gates Can Be Viewed as an Identity Matrix}

Consider the three-qubits situation, and suppose we want to apply the gate $U$ only to the third qubit. We would write it as below.

\begin{center}
\begin{quantikz}
    \lstick{qubit 1} & \qw & \qw\\begin{equation}
    0.3cm]
    \lstick{qubit 2} & \qw & \qw\\
    \lstick{qubit 3} & \gate{U} & \qw
\end{quantikz}
\end{center}
\vspace{0.5cm}

We can imagine that an identity gate is applied to all the empty qubit lines. Therefore, it can be rewrote as below.
\vspace{0.5cm}

\begin{center}
\begin{quantikz}
    \lstick{qubit 1} & \gate{I} & \qw\\
    \lstick{qubit 2} & \gate{I} & \qw\\
    \lstick{qubit 3} & \gate{U} & \qw
\end{quantikz}
\end{center}
\vspace{0.5cm}

These two quantum circuit expressions have the same meaning.
\subsubsection{Different Expressions about Entangled Input State and Product Input State}

\begin{definition}[Mention Again]
  A quantum state $|\psi\rangle \in\left(\mathbb{C}^2\right)^{\otimes q}$ is a \textbf{product state} if it can be expressed as a tensor product $\left|\psi_1\right\rangle \otimes \cdots \otimes\left|\psi_q\right\rangle$ of $q$ 1-qubit states. Otherwise, it is \textbf{entangled state}.
\end{definition}

\begin{property}[Mention Again]
    Let $A\in\mathbb{C}^{m \times m}, B\in \mathbb{C}^{n \times n}$ be linear transformations, $u\in \mathbb{C}^m, v\in \mathbb{C}^n$. The tensor product satisfies
  $(A \otimes B)(u \otimes v)=A u \otimes C v$.
\end{property}

If we have a product state $\ket{\psi}\otimes\ket{\phi}\otimes\ket{\xi}$ as input state. The only gate is $U$ that applied on the third qubit. Then we can rewrite the quantum circuit as below.
\vspace{0.5cm}

\begin{center}
\begin{quantikz}
    \lstick{qubit 1: $\ket{\psi}$} & \gate{I} & \qw \rstick{$I\ket{\psi}$}\\
    \lstick{qubit 2: $\ket{\phi}$} & \gate{I} & \qw \rstick{$I\ket{\phi}$}\\
    \lstick{qubit 3: $\ket{\xi}$} & \gate{U} & \qw \rstick{$U\ket{\xi}$}
\end{quantikz}
\end{center}
\vspace{0.5cm}

In this situation we can write the output state as $(I \otimes I \otimes U)(|\psi\rangle \otimes|\phi\rangle \otimes|\xi\rangle)$ or $|\psi\rangle \otimes|\phi\rangle \otimes U|\xi\rangle$.
But when the input state is entangled state, it cannot be factored as a product state such as $|\psi\rangle\otimes|\phi\rangle \otimes|\xi\rangle$. Thus, for a general entangled 3-qubits input state $\ket{\zeta}$, we can only express the quantum circuit as below.

\begin{center}
\begin{quantikz}
    \lstick[wires=3]{$\ket{\zeta}$} & \gate{I} & \qw \rstick[wires=3]{$(I \otimes I \otimes U)\ket{\zeta}$}\\
     & \gate{I} & \qw \\
     & \gate{U} & \qw 
\end{quantikz}
\end{center}
\vspace{0.5cm}

\subsubsection{Use an New $4\times4$ Operation to Express Two $2\times2$ Operations}
\begin{property}\label{property_2}
    $A$ and $B$ are two unitary matrices, where $A, B\in \mathbb{C}^{2\times2}$. $V:=A\otimes B$ is also an unitary matrix. 
\end{property}
\begin{proof}
$A$ and $B$ are two unitary matrices, so
\begin{equation}
    A=\left[\begin{array}{ll}
    a_{11} & a_{12} \\
    a_{21} & a_{22}
    \end{array}\right] \quad, B=\left[\begin{array}{ll}
    b_{11} & b_{12} \\
    b_{21} & b_{22}
    \end{array}\right].
\end{equation}
Then we have 
\begin{equation}
    V:=A \otimes B=\left[\begin{array}{ll}
a_{11} B & a_{12} B \\
a_{21} B & a_{22} B
\end{array}\right], 
V^*=\left[\begin{array}{ll}
a_{11}^* B^* & a_{21}^* B^* \\
a_{12}^* B^* & a_{22}^* B^*
\end{array}\right].
\end{equation}

\begin{equation}
    V V^*=\left[\begin{array}{ll}
a_{11} a_{11}^* B B^*+a_{12} a_{12}^* B B^* & a_{11} a_{21}^* B B^*+a_{12} a_{22}^* B B^* \\
a_{21} a_{11}^* B B^*+a_{22} a_{12}^* B B^* & a_{21} a_{21}^* B B^*+a_{22} a_{22}^* B B^*
\end{array}\right].
\end{equation}

% & A^*=\left[\begin{array}{ll}
% a_{11}^* & a_{21}^* \\
% a_{12}^* & a_{22}^*
% \end{array}\right] \quad \\

$A$ is unitary, so
\begin{equation}
    A A^*=\left[\begin{array}{ll}
a_{11} a_{11}^*+a_{12} a_{12}^* & a_{11} a_{21} *+a_{12} a_{22} * \\
a_{21} a_{11}^*+a_{22} a_{12}^* & a_{21} a_{21}^*+a_{22} a_{22}^*
\end{array}\right]=\left[\begin{array}{ll}
1 & 0 \\
0 & 1
\end{array}\right].
\end{equation}

According to above, we can get
\begin{equation}
    a_{11} a_{11}^*+a_{12} a_{12}^*=a_{21} a_{21}^*+a_{22} a_{22}^*=1,
\end{equation}
\begin{equation}
    a_{11} a_{21}^*+a_{12} a_{22}^*=a_{21}a_{11}^*+a_{22} a_{12}{ }^*=0 .
\end{equation}

$B$ is also an unitary matrix, so 
\begin{equation}
    B B^*=\left[\begin{array}{ll}
1 & 0 \\
0 & 1
\end{array}\right].
\end{equation}

Finally we can get 
\begin{equation}
    V V^*=\left[\begin{array}{ll}
(a_{11} a_{11}^*+a_{12} a_{12}^*)B B^* & (a_{11} a_{21}^*+a_{12} a_{22}^*)B B^* \\
(a_{21} a_{11}^*+a_{22} a_{12}^*)B B^* & (a_{21} a_{21}^* +a_{22} a_{22}^*)B B^*
\end{array}\right]\\
=\left[\begin{array}{cccc}
1 & 0 & 0 & 0 \\
0 & 1 & 0 & 0 \\
0 & 0 & 1 & 0 \\
0 & 0 & 0 & 1 \\
\end{array}\right].
\end{equation}

\end{proof}

Consider the situation that we apply the operation $A\in\mathbb{C}^{2\times2}$ on the first qubit and apply the operation $B\in\mathbb{C}^{2\times2}$ on the second qubit. According to \textbf{Property} \ref{property_2} and \textbf{Property} \ref{property_1}, $V:=A\otimes B$ is also an unitary matrix. If the input state is a product state $|\xi\rangle=|\psi\rangle\otimes|\phi\rangle$, then we will get the output state $V(|\psi\rangle\otimes|\phi\rangle)=(V_1\otimes V_2)(|\psi\rangle\otimes|\phi\rangle)=(A|\psi\rangle)\otimes(B|\phi\rangle)$, where $V=A\otimes B, V\in \mathbb{C}^{4\times4}, A, B\in \mathbb{C}^{2\times2}$. The first qubit of the output state is $A|\psi\rangle$ and the second qubit of the output state is $B|\phi\rangle$.
\vspace{0.5cm}

\begin{center}
\begin{quantikz}
    \lstick{\ket{\psi}} & \gate[wires=2]{V} & \qw \rstick{$A$\ket{\psi}}\\
    \lstick{\ket{\phi}} & \qw & \qw \rstick{$B$\ket{\phi}}
\end{quantikz}
\end{center}
\vspace{0.5cm}

But when the input state is entangled state, we can just use $|\xi\rangle$ to express the input state and use $V|\psi \rangle$ to express the output state. 
\vspace{0.5cm}

\begin{center}
\begin{quantikz}
    \lstick[wires=2]{$\ket{\xi}$} & \gate[wires=2]{V} & \qw\rstick[wires=2]{$V\ket{\xi}$}\\
    &  & \qw 
\end{quantikz}
\end{center}
\vspace{0.5cm}

\subsubsection{Measurement and NOT Gate}
In a classical computer we can simply read the state of the bits by measuring whether the result is high voltage or low voltage. High voltage represents 1 and low voltage represents 0, whereas in a quantum computer we do not have direct, unrestricted access to the quantum state. Consider a 1-qubit input state $|\psi\rangle=\alpha|0\rangle+\beta|1\rangle$, where $|\alpha|^2+|\beta|^2=1$, $\alpha, \beta\in \mathbb{C}$. Information in this quantum state can only be gathered through a \textbf{measurement gate}, indicated in the circuit diagram as below.
\vspace{0.5cm}

\begin{center}
\begin{quantikz}
    \lstick{$\ket{\psi}$} & \meter{} & \qw \rstick{$\ket{0}$ or $\ket{1}$}
\end{quantikz}
\end{center}
\vspace{0.5cm}

In fact, we will observe that the output state is $|0\rangle$ with probability $|\alpha|^2$ and is $|1\rangle$ with probability $|\beta|^2$. In mathematical form it means
\begin{align}
    \operatorname{Prob}(\text{output}=\ket{0})=|\alpha|^2,\\
    \operatorname{Prob}(\text{output}=\ket{1})=|\beta|^2.
\end{align}  

The \textbf{NOT gate} in quantum computation takes the state
$
\alpha|0\rangle+\beta|1\rangle
$
to the corresponding state in which the role of $|0\rangle$ and $|1\rangle$ have been interchanged,
\end{equation}
\alpha|1\rangle+\beta|0\rangle
\begin{equation}
    where $|\alpha|+|\beta|^2=1$.

There is a convenient way of representing the quantum NOT gate in matrix form as
\end{equation}
X =\left[\begin{array}{ll}
0 & 1 \\
1 & 0
\end{array}\right] \text {. }
\begin{equation}
    % (The notation $X$ for the quantum NOT is used for historical reasons.) If the quantum state $\alpha|0\rangle+\beta|1\rangle$ is written in a vector notation as
%
\end{equation}
% \left[\begin{array}{l}
% \alpha \\
% \beta
% \end{array}\right],
% \begin{equation}
    % with the top entry corresponding to the amplitude for $|0\rangle$ and the bottom entry the amplitude for $|1\rangle$, 

The input state is $|\psi\rangle=\alpha|0\rangle+\beta|1\rangle$ and the output state will be
\end{equation}
X\left[\begin{array}{l}
\alpha \\
\beta
\end{array}\right]=\left[\begin{array}{l}
\beta \\
\alpha
\end{array}\right]=\beta\ket{0}+\alpha\ket{1}.
\begin{equation}
\vspace{0.5cm}

\begin{center}
    \begin{quantikz}
    \lstick{$\alpha\ket{0}+\beta\ket{1}$} & \gate{X} & \rstick{$\beta\ket{0}+\alpha\ket{1}$}\qw
    \end{quantikz}
\end{center}
