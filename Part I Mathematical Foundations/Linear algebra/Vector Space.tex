A good understanding of quantum mechanics is based upon a solid grasp of linear algebra.

The vector space of most interest to us is $\mathbf{C}^{n}$, the space of all $n$-tuples of complex numbers, $\left(z_{1}, \ldots, z_{n}\right)$. The elements of a vector space are called vectors, and we will sometimes use the column matrix notation
\begin{equation}
    \left[\begin{array}{c}
z_{1} \\
\vdots \\
z_{n}
\end{array}\right]
\end{equation}
to indicate a vector. For convenience, we use the notation $\left(z_{1}, \ldots, z_{n}\right)$ to denote a column matrix with entries $z_{1}, \ldots, z_{n}$. 

There is an addition operation defined which takes pairs of vectors to other vectors. In $\mathbf{C}^{n}$ the addition operation for vectors is defined by
\begin{equation}
    \left[\begin{array}{c}
z_{1} \\
\vdots \\
z_{n}
\end{array}\right]+\left[\begin{array}{c}
z_{1}^{\prime} \\
\vdots \\
z_{n}^{\prime}
\end{array}\right] \equiv\left[\begin{array}{c}
z_{1}+z_{1}^{\prime} \\
\vdots \\
z_{n}+z_{n}^{\prime}
\end{array}\right]
\end{equation}
where the addition operations on the right are just ordinary additions of complex numbers. Furthermore, in a vector space there is a scalar multiplication operation. In $\mathbf{C}^{n}$ this is defined by
\begin{equation}
    z\left[\begin{array}{c}
z_{1} \\
\vdots \\
z_{n}
\end{array}\right] \equiv\left[\begin{array}{c}
z z_{1} \\
\vdots \\
z z_{n}
\end{array}\right]
\end{equation}
where $z$ is a scalar, that is, a complex number\footnote{Physicists sometimes refer to complex numbers as $c$-numbers.}, and the multiplications on the right are ordinary multiplication of complex numbers. A vector subspace of a vector space $V$ is a subset $W$ of $V$ such that $W$ is also a vector space, that is, $W$ must be closed under scalar multiplication and addition.

We will use the \textit{standard} notation of quantum mechanics for linear algebraic concepts, see Figure \ref{tab:summary_AL}. For a vector in a vector space is the following:
\begin{equation}
    |\psi\rangle .
\end{equation}
The notation $|\cdot\rangle$, called "ket", is used to indicate that the object is a (column) vector, and $\psi$ is a label for the vector. The entire object $|\psi\rangle$ is pronounced  as "ket psi". Any label is valid, although we prefer to use simple labels like $\psi$ and $\varphi$.

\begin{remark} %zero vector $0$ and the special vector $|0\rangle$
    A vector space also contains a special zero vector, which we denote by 0. In $\mathbf{C}^{n}$ the zero element is $0=(0,0, \ldots, 0)$. Note that we do not use the ket notation for the zero vector - it is the only exception we shall make. Because, it is conventional to use the notation $|0\rangle$ to mean some special vector that is not zero vector.
\end{remark}

\begin{table}[]
    \centering
    \begin{tabular}{c|l}
        \hline\hline
        Notation & Description \\
        \hline
        $z^{*}$ & Complex conjugate of the complex number $z$. $(1+i)^{*}=1-i$\\
        $|\psi\rangle$& Vector. Also known as a ket. \\
        $\langle\psi|$& Vector dual to $|\psi\rangle$. Also known as a bra. \\
        $\langle\varphi | \psi\rangle$& Inner product between the vectors $|\varphi\rangle$ and $|\psi\rangle$. \\
        $|\varphi\rangle \otimes|\psi\rangle$& Tensor product of $|\varphi\rangle$ and $|\psi\rangle$. \\
        $|\varphi\rangle|\psi\rangle$& Abbreviated notation for tensor product of $|\varphi\rangle$ and $|\psi\rangle$. \\
        $A^{*}$& Complex conjugate of the $A$ matrix. \\
        $A^{T}$& Transpose of the $A$ matrix. \\
         $A^{\dagger}$& Hermitian conjugate or adjoint of the $A$ matrix, $A^{\dagger}=\left(A^{T}\right)^{*}$. \\
         & $\left[\begin{array}{cc}a & b \\ c & d\end{array}\right]^{\dagger}=\left[\begin{array}{cc}a^{*} & c^{*} \\ b^{*} & d^{*}\end{array}\right]$. \\
        $\langle\varphi|A| \psi\rangle$ & Inner product between $|\varphi\rangle$ and $A|\psi\rangle$. \\
         & Equivalently, inner product between $A^{\dagger}|\varphi\rangle$ and $|\psi\rangle$. \\
        \hline\hline
    \end{tabular}
\caption{Summary of some standard quantum mechanical notation for notions from linear algebra. This style of notation is known as the Dirac notation.}
\label{tab:summary_AL}
\end{table}