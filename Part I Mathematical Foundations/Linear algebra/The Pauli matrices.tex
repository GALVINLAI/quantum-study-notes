%原2.1.3

Four extremely useful matrices which we shall often have occasion to use are the Pauli matrices. These are 2 by 2 matrices, which go by a variety of notations. The matrices, and their corresponding notations, are depicted in Figure 2.2. 

The Pauli matrices are so useful in the study of quantum computation and quantum information.

\begin{equation}
\begin{aligned}
\sigma_{0} \equiv I \equiv\left[\begin{array}{rr}
1 & 0 \\
0 & 1
\end{array}\right] & \sigma_{1} \equiv \sigma_{x} \equiv X \equiv\left[\begin{array}{rr}
0 & 1 \\
1 & 0
\end{array}\right] \\
\sigma_{2} \equiv \sigma_{y} \equiv Y \equiv\left[\begin{array}{rr}
0 & -i \\
i & 0
\end{array}\right] & \sigma_{3} \equiv \sigma_{z} \equiv Z \equiv\left[\begin{array}{rr}
1 & 0 \\
0 & -1
\end{array}\right]
\end{aligned}
\end{equation}

Figure 2.2. The Pauli matrices. Sometimes $I$ is omitted from the list with just $X, Y$ and $Z$ known as the Pauli matrices.