
\begin{definition}
    An eigenvector of a linear operator $A$ on a vector space is a non-zero vector $|v\rangle$ such that $A|v\rangle=v|v\rangle$, where $v$ is a complex number known as the eigenvalue of $A$ corresponding to $|v\rangle$. 
\end{definition}

\begin{remark}
    It will often be convenient to use the notation $v$ both as a label for the eigenvector, and to represent the eigenvalue. 
\end{remark}

% We assume that you are familiar with the elementary properties of eigenvalues and eigenvectors - in particular, how to find them, via the characteristic equation. 

The characteristic function is defined to be $c(\lambda) \equiv \operatorname{det}|A-\lambda I|$, where det is the determinant function for matrices; it can be shown that the characteristic function depends only upon the operator $A$, and not on the specific matrix representation used for $A$. 

The solutions of the characteristic equation $c(\lambda)=0$ are the eigenvalues of the operator $A$. By the fundamental theorem of algebra, every polynomial has at least one complex root, so every operator $A$ has at least one eigenvalue, and a corresponding eigenvector. 

The eigenspace corresponding to an eigenvalue $v$ is the set of vectors which have eigenvalue $v$. It is a vector subspace of the vector space on which $A$ acts.

\begin{definition}[diagonal representation]
    A \textit{diagonal representation}\footnote{Diagonal representations are sometimes also known as orthonormal decompositions.} for an operator $A$ on a vector space $V$ is a representation 
    \begin{equation}
    A=\sum_{i} \lambda_{i}|i\rangle\langle i|,
\end{equation}
where the vectors $|i\rangle$ form an orthonormal set of eigenvectors for $A$, with corresponding eigenvalues $\lambda_{i}$. An operator is said to be \textit{diagonalizable} if it has a diagonal representation. 
\end{definition}

In the next section we will find a simple set of necessary and sufficient conditions for an operator on a Hilbert space to be diagonalizable. 

\begin{example}
    As an example of a diagonal representation, note that the Pauli $Z$ matrix may be written
\begin{equation}
    Z=\left[\begin{array}{rr}
1 & 0 \\
0 & -1
\end{array}\right]=|0\rangle\langle 0|-| 1\rangle\langle 1|,
\end{equation}
where the matrix representation is with respect to orthonormal vectors $|0\rangle$ and $|1\rangle$, respectively. 
\end{example}

When an eigenspace is more than one dimensional we say that it is \textit{degenerate}. For example, the matrix $A$ defined by
\begin{equation}
    A \equiv\left[\begin{array}{lll}
2 & 0 & 0 \\
0 & 2 & 0 \\
0 & 0 & 0
\end{array}\right]
\end{equation}
has a two-dimensional eigenspace corresponding to the eigenvalue 2 . The eigenvectors $(1,0,0)$ and $(0,1,0)$ are said to be degenerate because they are linearly independent eigenvectors of $A$ with the same eigenvalue.

\begin{exercise}
    Exercise 2.11: (Eigendecomposition of the Pauli matrices) Find the eigenvectors, eigenvalues, and diagonal representations of the Pauli matrices $X, Y$, and $Z$.
\end{exercise}

\begin{exercise}
    Exercise 2.12: Prove that the matrix
\begin{equation}
    \left[\begin{array}{ll}
1 & 0 \\
1 & 1
\end{array}\right]
\end{equation}
is not diagonalizable.
\end{exercise}

