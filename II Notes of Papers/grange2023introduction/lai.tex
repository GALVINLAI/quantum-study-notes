Variational Quantum Algorithms are heuristic algorithms that alternate between a quantum circuit and a classical optimizer. They tackle optimization problems of the form
$$
\min _{x \in\{0,1\}^n} f(x),
$$
where $f$ is any function defined on $\{0,1\}^n$. VQAs are of great interest to the quantum information

Example 8 (Generic one-qubit gate). A generic one-qubit gate $U=\left(\begin{array}{ll}a & b \\ c & d\end{array}\right) \in \mathcal{M}_2(\mathbb{C})$ satisfies
$$
\left\{\begin{array}{l}
|a|^2+|b|^2=1 \\
a \bar{c}+b \bar{d}=0 \\
\bar{a} c+\bar{b} d=0 \\
|c|^2+|d|^2=1
\end{array}\right.
$$

Its application on a qubit $|q\rangle=q_0|0\rangle+q_1|1\rangle$ is:
$$
U|q\rangle=\left(\begin{array}{ll}
a & b \\
c & d
\end{array}\right)\binom{q_0}{q_1}=\binom{a q_0+b q_1}{c q_0+d q_1}=\left(a q_0+b q_1\right)|0\rangle+\left(c q_0+d q_1\right)|1\rangle=\left|q^{\prime}\right\rangle
$$
