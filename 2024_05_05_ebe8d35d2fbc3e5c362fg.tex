\documentclass[10pt]{article}
\usepackage[utf8]{inputenc}
\usepackage[T1]{fontenc}
\usepackage{amsmath}
\usepackage{amsfonts}
\usepackage{amssymb}
\usepackage[version=4]{mhchem}
\usepackage{stmaryrd}

\title{2.2 The postulates of quantum mechanics }

\author{}
\date{}


\begin{document}
\maketitle
\begin{abstract}
All understanding begins with our not accepting the world as it appears. - Alan Kay
\end{abstract}

The most incomprehensible thing about the world is that it is comprehensible. - Albert Einstein

Quantum mechanics is a mathematical framework for the development of physical theories. On its own quantum mechanics doesn't tell you what laws a physical system must obey, but it does provide a mathematical and conceptual framework for the development of such laws. In the next few sections we give a complete description of the basic postulates of quantum mechanics. These postulates provide a connection between the physical world and the mathematical formalism of quantum mechanics.

The postulates of quantum mechanics were derived after a long process of trial and (mostly) error, which involved a considerable amount of guessing and fumbling by the originators of the theory. Don't be surprised if the motivation for the postulates is not always clear; even to experts the basic postulates of quantum mechanics appear surprising. What you should expect to gain in the next few sections is a good working grasp of the postulates - how to apply them, and when.

\subsection*{2.2.1 State space}
The first postulate of quantum mechanics sets up the arena in which quantum mechanics takes place. The arena is our familiar friend from linear algebra, Hilbert space.

Postulate 1: Associated to any isolated physical system is a complex vector space with inner product (that is, a Hilbert space) known as the state space of the system. The system is completely described by its state vector, which is a unit vector in the system's state space.

Quantum mechanics does not tell us, for a given physical system, what the state space of that system is, nor does it tell us what the state vector of the system is. Figuring that out for a specific system is a difficult problem for which physicists have developed many intricate and beautiful rules. For example, there is the wonderful theory of quantum electrodynamics (often known as QED), which describes how atoms and light interact. One aspect of QED is that it tells us what state spaces to use to give quantum descriptions of atoms and light. We won't be much concerned with the intricacies of theories like QED (except in so far as they apply to physical realizations, in Chapter 7), as we are mostly interested in the general framework provided by quantum mechanics. For our purposes it will be sufficient to make some very simple (and reasonable) assumptions about the state spaces of the systems we are interested in, and stick with those assumptions.

The simplest quantum mechanical system, and the system which we will be most concerned with, is the qubit. A qubit has a two-dimensional state space. Suppose $|0\rangle$ and $|1\rangle$ form an orthonormal basis for that state space. Then an arbitrary state vector in the state space can be written


\begin{equation*}
|\psi\rangle=a|0\rangle+b|1\rangle \tag{2.82}
\end{equation*}


where $a$ and $b$ are complex numbers. The condition that $|\psi\rangle$ be a unit vector, $\langle\psi \mid \psi\rangle=1$, is therefore equivalent to $|a|^{2}+|b|^{2}=1$. The condition $\langle\psi \mid \psi\rangle=1$ is often known as the normalization condition for state vectors.

We will take the qubit as our fundamental quantum mechanical system. Later, in Chapter 7, we will see that there are real physical systems which may be described in terms of qubits. For now, though, it is sufficient to think of qubits in abstract terms, without reference to a specific realization. Our discussions of qubits will always be referred to some orthonormal set of basis vectors, $|0\rangle$ and $|1\rangle$, which should be thought of as being fixed in advance. Intuitively, the states $|0\rangle$ and $|1\rangle$ are analogous to the two values 0 and 1 which a bit may take. The way a qubit differs from a bit is that superpositions of these two states, of the form $a|0\rangle+b|1\rangle$, can also exist, in which it is not possible to say that the qubit is definitely in the state $|0\rangle$, or definitely in the state $|1\rangle$.

We conclude with some useful terminology which is often used in connection with the description of quantum states. We say that any linear combination $\sum_{i} \alpha_{i}\left|\psi_{i}\right\rangle$ is a superposition of the states $\left|\psi_{i}\right\rangle$ with amplitude $\alpha_{i}$ for the state $\left|\psi_{i}\right\rangle$. So, for example, the state


\begin{equation*}
\frac{|0\rangle-|1\rangle}{\sqrt{2}} \tag{2.83}
\end{equation*}


is a superposition of the states $|0\rangle$ and $|1\rangle$ with amplitude $1 / \sqrt{2}$ for the state $|0\rangle$, and amplitude $-1 / \sqrt{2}$ for the state $|1\rangle$.

\subsection*{2.2.2 Evolution}
How does the state, $|\psi\rangle$, of a quantum mechanical system change with time? The following postulate gives a prescription for the description of such state changes.

Postulate 2: The evolution of a closed quantum system is described by a unitary transformation. That is, the state $|\psi\rangle$ of the system at time $t_{1}$ is related to the state $\left|\psi^{\prime}\right\rangle$ of the system at time $t_{2}$ by a unitary operator $U$ which depends only on the times $t_{1}$ and $t_{2}$,


\begin{equation*}
\left|\psi^{\prime}\right\rangle=U|\psi\rangle \tag{2.84}
\end{equation*}


Just as quantum mechanics does not tell us the state space or quantum state of a particular quantum system, it does not tell us which unitary operators $U$ describe realworld quantum dynamics. Quantum mechanics merely assures us that the evolution of any closed quantum system may be described in such a way. An obvious question to ask is: what unitary operators are natural to consider? In the case of single qubits, it turns out that any unitary operator at all can be realized in realistic systems.

Let's look at a few examples of unitary operators on a single qubit which are important in quantum computation and quantum information. We have already seen several examples of such unitary operators - the Pauli matrices, defined in Section 2.1.3, and the quantum gates described in Chapter 1. As remarked in Section 1.3.1, the $X$ matrix is often known as the quantum NOT gate, by analogy to the classical NOT gate. The $X$ and $Z$ Pauli matrices are also sometimes referred to as the bit flip and phase flip matrices: the $X$ matrix takes $|0\rangle$ to $|1\rangle$, and $|1\rangle$ to $|0\rangle$, thus earning the name bit flip; and the $Z$ matrix leaves $|0\rangle$ invariant, and takes $|1\rangle$ to $-|1\rangle$, with the extra factor of -1 added known as a phase factor, thus justifying the term phase flip. We will not use the term phase flip for\\
$Z$ very often, since it is easily confused with the phase gate to be defined in Chapter 4. (Section 2.2.7 contains more discussion of the many uses of the term 'phase'.)

Another interesting unitary operator is the Hadamard gate, which we denote $H$. This has the action $H|0\rangle \equiv(|0\rangle+|1\rangle) / \sqrt{2}, H|1\rangle \equiv(|0\rangle-|1\rangle) / \sqrt{2}$, and corresponding matrix representation

\[
H=\frac{1}{\sqrt{2}}\left[\begin{array}{cc}
1 & 1  \tag{2.85}\\
1 & -1
\end{array}\right]
\]

Exercise 2.51: Verify that the Hadamard gate $H$ is unitary.

Exercise 2.52: Verify that $H^{2}=I$.

Exercise 2.53: What are the eigenvalues and eigenvectors of $H$ ?

Postulate 2 requires that the system being described be closed. That is, it is not interacting in any way with other systems. In reality, of course, all systems (except the Universe as a whole) interact at least somewhat with other systems. Nevertheless, there are interesting systems which can be described to a good approximation as being closed, and which are described by unitary evolution to some good approximation. Furthermore, at least in principle every open system can be described as part of a larger closed system (the Universe) which is undergoing unitary evolution. Later, we'll introduce more tools which allow us to describe systems which are not closed, but for now we'll continue with the description of the evolution of closed systems.

Postulate 2 describes how the quantum states of a closed quantum system at two different times are related. A more refined version of this postulate can be given which describes the evolution of a quantum system in continuous time. From this more refined postulate we will recover Postulate 2. Before we state the revised postulate, it is worth pointing out two things. First, a notational remark. The operator $H$ appearing in the following discussion is not the same as the Hadamard operator, which we just introduced. Second, the following postulate makes use of the apparatus of differential equations. Readers with little background in the study of differential equations should be reassured that they will not be necessary for much of the book, with the exception of parts of Chapter 7, on real physical implementations of quantum information processing.

Postulate $2^{\prime}$ : The time evolution of the state of a closed quantum system is described by the Schrödinger equation,


\begin{equation*}
i \hbar \frac{d|\psi\rangle}{d t}=H|\psi\rangle \tag{2.86}
\end{equation*}


In this equation, $\hbar$ is a physical constant known as Planck's constant whose value must be experimentally determined. The exact value is not important to us. In practice, it is common to absorb the factor $\hbar$ into $H$, effectively setting $\hbar=1$. $H$ is a fixed Hermitian operator known as the Hamiltonian of the closed system.

If we know the Hamiltonian of a system, then (together with a knowledge of $\hbar$ ) we understand its dynamics completely, at least in principle. In general figuring out the Hamiltonian needed to describe a particular physical system is a very difficult problem - much of twentieth century physics has been concerned with this problem - which requires substantial input from experiment in order to be answered. From our point of\\
view this is a problem of detail to be addressed by physical theories built within the framework of quantum mechanics - what Hamiltonian do we need to describe atoms in such-and-such a configuration - and is not a question that needs to be addressed by the theory of quantum mechanics itself. Most of the time in our discussion of quantum computation and quantum information we won't need to discuss Hamiltonians, and when we do, we will usually just posit that some matrix is the Hamiltonian as a starting point, and proceed from there, without attempting to justify the use of that Hamiltonian.

Because the Hamiltonian is a Hermitian operator it has a spectral decomposition


\begin{equation*}
H=\sum_{E} E|E\rangle\langle E| \tag{2.87}
\end{equation*}


with eigenvalues $E$ and corresponding normalized eigenvectors $|E\rangle$. The states $|E\rangle$ are conventionally referred to as energy eigenstates, or sometimes as stationary states, and $E$ is the energy of the state $|E\rangle$. The lowest energy is known as the ground state energy for the system, and the corresponding energy eigenstate (or eigenspace) is known as the ground state. The reason the states $|E\rangle$ are sometimes known as stationary states is because their only change in time is to acquire an overall numerical factor,


\begin{equation*}
|E\rangle \rightarrow \exp (-i E t / \hbar)|E\rangle \tag{2.88}
\end{equation*}


As an example, suppose a single qubit has Hamiltonian


\begin{equation*}
H=\hbar \omega X \tag{2.89}
\end{equation*}


In this equation $\omega$ is a parameter that, in practice, needs to be experimentally determined. We won't worry about the parameter overly much here - the point is to give you a feel for the sort of Hamiltonians that are sometimes written down in the study of quantum computation and quantum information. The energy eigenstates of this Hamiltonian are obviously the same as the eigenstates of $X$, namely $(|0\rangle+|1\rangle) / \sqrt{2}$ and $(|0\rangle-|1\rangle) / \sqrt{2}$, with corresponding energies $\hbar \omega$ and $-\hbar \omega$. The ground state is therefore $(|0\rangle-|1\rangle) / \sqrt{2}$, and the ground state energy is $-\hbar \omega$.

What is the connection between the Hamiltonian picture of dynamics, Postulate $2^{\prime}$, and the unitary operator picture, Postulate 2? The answer is provided by writing down the solution to Schrödinger's equation, which is easily verified to be:


\begin{equation*}
\left|\psi\left(t_{2}\right)\right\rangle=\exp \left[\frac{-i H\left(t_{2}-t_{1}\right)}{\hbar}\right]\left|\psi\left(t_{1}\right)\right\rangle=U\left(t_{1}, t_{2}\right)\left|\psi\left(t_{1}\right)\right\rangle \tag{2.90}
\end{equation*}


where we define


\begin{equation*}
U\left(t_{1}, t_{2}\right) \equiv \exp \left[\frac{-i H\left(t_{2}-t_{1}\right)}{\hbar}\right] \tag{2.91}
\end{equation*}


You will show in the exercises that this operator is unitary, and furthermore, that any unitary operator $U$ can be realized in the form $U=\exp (i K)$ for some Hermitian operator $K$. There is therefore a one-to-one correspondence between the discrete-time description of dynamics using unitary operators, and the continuous time description using Hamiltonians. For most of the book we use the unitary formulation of quantum dynamics.

Exercise 2.54: Suppose $A$ and $B$ are commuting Hermitian operators. Prove that $\exp (A) \exp (B)=\exp (A+B) .($ Hint: Use the results of Section 2.1.9.)

Exercise 2.55: Prove that $U\left(t_{1}, t_{2}\right)$ defined in Equation (2.91) is unitary.

Exercise 2.56: Use the spectral decomposition to show that $K \equiv-i \log (U)$ is Hermitian for any unitary $U$, and thus $U=\exp (i K)$ for some Hermitian $K$.

In quantum computation and quantum information we often speak of applying a unitary operator to a particular quantum system. For example, in the context of quantum circuits we may speak of applying the unitary gate $X$ to a single qubit. Doesn't this contradict what we said earlier, about unitary operators describing the evolution of a closed quantum system? After all, if we are 'applying' a unitary operator, then that implies that there is an external 'we' who is interacting with the quantum system, and the system is not closed.

An example of this occurs when a laser is focused on an atom. After a lot of thought and hard work it is possible to write down a Hamiltonian describing the total atomlaser system. The interesting thing is that when we write down the Hamiltonian for the atom-laser system and consider the effects on the atom alone, the behavior of the state vector of the atom turns out to be almost but not quite perfectly described by another Hamiltonian, the atomic Hamiltonian. The atomic Hamiltonian contains terms related to laser intensity, and other parameters of the laser, which we can vary at will. It is as if the evolution of the atom were being described by a Hamiltonian which we can vary at will, despite the atom not being a closed system.

More generally, for many systems like this it turns out to be possible to write down a time-varying Hamiltonian for a quantum system, in which the Hamiltonian for the system is not a constant, but varies according to some parameters which are under an experimentalist's control, and which may be changed during the course of an experiment. The system is not, therefore, closed, but it does evolve according to Schrödinger's equation with a time-varying Hamiltonian, to some good approximation.

The upshot is that to begin we will often describe the evolution of quantum systems even systems which aren't closed - using unitary operators. The main exception to this, quantum measurement, will be described in the next section. Later on we will investigate in more detail possible deviations from unitary evolution due to the interaction with other systems, and understand more precisely the dynamics of realistic quantum systems.

\subsection*{2.2.3 Quantum measurement}
We postulated that closed quantum systems evolve according to unitary evolution. The evolution of systems which don't interact with the rest of the world is all very well, but there must also be times when the experimentalist and their experimental equipment an external physical system in other words - observes the system to find out what is going on inside the system, an interaction which makes the system no longer closed, and thus not necessarily subject to unitary evolution. To explain what happens when this is done, we introduce Postulate 3, which provides a means for describing the effects of measurements on quantum systems.

Postulate 3: Quantum measurements are described by a collection $\left\{M_{m}\right\}$ of measurement operators. These are operators acting on the state space of the system being measured. The index $m$ refers to the measurement outcomes that may occur in the experiment. If the state of the quantum system is $|\psi\rangle$ immediately before the measurement then the probability that result $m$ occurs is\\
given by


\begin{equation*}
p(m)=\left\langle\psi\left|M_{m}^{\dagger} M_{m}\right| \psi\right\rangle \tag{2.92}
\end{equation*}


and the state of the system after the measurement is


\begin{equation*}
\frac{M_{m}|\psi\rangle}{\sqrt{\left\langle\psi\left|M_{m}^{\dagger} M_{m}\right| \psi\right\rangle}} \tag{2.93}
\end{equation*}


The measurement operators satisfy the completeness equation,


\begin{equation*}
\sum_{m} M_{m}^{\dagger} M_{m}=I \tag{2.94}
\end{equation*}


The completeness equation expresses the fact that probabilities sum to one:


\begin{equation*}
1=\sum_{m} p(m)=\sum_{m}\left\langle\psi\left|M_{m}^{\dagger} M_{m}\right| \psi\right\rangle \tag{2.95}
\end{equation*}


This equation being satisfied for all $|\psi\rangle$ is equivalent to the completeness equation. However, the completeness equation is much easier to check directly, so that's why it appears in the statement of the postulate.

A simple but important example of a measurement is the measurement of a qubit in the computational basis. This is a measurement on a single qubit with two outcomes defined by the two measurement operators $M_{0}=|0\rangle\left\langle 0\left|, M_{1}=\right| 1\right\rangle\langle 1|$. Observe that each measurement operator is Hermitian, and that $M_{0}^{2}=M_{0}, M_{1}^{2}=M_{1}$. Thus the completeness relation is obeyed, $I=M_{0}^{\dagger} M_{0}+M_{1}^{\dagger} M_{1}=M_{0}+M_{1}$. Suppose the state being measured is $|\psi\rangle=a|0\rangle+b|1\rangle$. Then the probability of obtaining measurement outcome 0 is


\begin{equation*}
p(0)=\left\langle\psi\left|M_{0}^{\dagger} M_{0}\right| \psi\right\rangle=\left\langle\psi\left|M_{0}\right| \psi\right\rangle=|a|^{2} \tag{2.96}
\end{equation*}


Similarly, the probability of obtaining the measurement outcome 1 is $p(1)=|b|^{2}$. The state after measurement in the two cases is therefore


\begin{align*}
& \frac{M_{0}|\psi\rangle}{|a|}=\frac{a}{|a|}|0\rangle  \tag{2.97}\\
& \frac{M_{1}|\psi\rangle}{|b|}=\frac{b}{|b|}|1\rangle . \tag{2.98}
\end{align*}


We will see in Section 2.2 .7 that multipliers like $a /|a|$, which have modulus one, can effectively be ignored, so the two post-measurement states are effectively $|0\rangle$ and $|1\rangle$, just as described in Chapter 1.

The status of Postulate 3 as a fundamental postulate intrigues many people. Measuring devices are quantum mechanical systems, so the quantum system being measured and the measuring device together are part of a larger, isolated, quantum mechanical system. (It may be necessary to include quantum systems other than the system being measured and the measuring device to obtain a completely isolated system, but the point is that this can be done.) According to Postulate 2, the evolution of this larger isolated system can be described by a unitary evolution. Might it be possible to derive Postulate 3 as a consequence of this picture? Despite considerable investigation along these lines there is still disagreement between physicists about whether or not this is possible. We, however, are going to take the very pragmatic approach that in practice it is clear when to apply

Postulate 2 and when to apply Postulate 3 , and not worry about deriving one postulate from the other.

Over the next few sections we apply Postulate 3 to several elementary but important measurement scenarios. Section 2.2.4 examines the problem of distinguishing a set of quantum states. Section 2.2.5 explains a special case of Postulate 3, the projective or von Neumann measurements. Section 2.2.6 explains another special case of Postulate 3, known as $P O V M$ measurements. Many introductions to quantum mechanics only discuss projective measurements, omitting a full discussion of Postulate 3 or of POVM elements. For this reason we have included Box 2.5 on page 91 which comments on the relationship between the different classes of measurement we describe.

\section*{Exercise 2.57: (Cascaded measurements are single measurements) Suppose}
$\left\{L_{l}\right\}$ and $\left\{M_{m}\right\}$ are two sets of measurement operators. Show that a measurement defined by the measurement operators $\left\{L_{l}\right\}$ followed by a measurement defined by the measurement operators $\left\{M_{m}\right\}$ is physically equivalent to a single measurement defined by measurement operators $\left\{N_{l m}\right\}$ with the representation $N_{l m} \equiv M_{m} L_{l}$.

\subsection*{2.2.4 Distinguishing quantum states}
An important application of Postulate 3 is to the problem of distinguishing quantum states. In the classical world, distinct states of an object are usually distinguishable, at least in principle. For example, we can always identify whether a coin has landed heads or tails, at least in the ideal limit. Quantum mechanically, the situation is more complicated. In Section 1.6 we gave a plausible argument that non-orthogonal quantum states cannot be distinguished. With Postulate 3 as a firm foundation we can now give a much more convincing demonstration of this fact.

Distinguishability, like many ideas in quantum computation and quantum information, is most easily understood using the metaphor of a game involving two parties, Alice and Bob. Alice chooses a state $\left|\psi_{i}\right\rangle(1 \leq i \leq n)$ from some fixed set of states known to both parties. She gives the state $\left|\psi_{i}\right\rangle$ to Bob, whose task it is to identify the index $i$ of the state Alice has given him.

Suppose the states $\left|\psi_{i}\right\rangle$ are orthonormal. Then Bob can do a quantum measurement to distinguish these states, using the following procedure. Define measurement operators $M_{i} \equiv\left|\psi_{i}\right\rangle\left\langle\psi_{i}\right|$, one for each possible index $i$, and an additional measurement operator $M_{0}$ defined as the positive square root of the positive operator $I-\sum_{i \neq 0}\left|\psi_{i}\right\rangle\left\langle\psi_{i}\right|$. These operators satisfy the completeness relation, and if the state $\left|\psi_{i}\right\rangle$ is prepared then $p(i)=\left\langle\psi_{i}\left|M_{i}\right| \psi_{i}\right\rangle=1$, so the result $i$ occurs with certainty. Thus, it is possible to reliably distinguish the orthonormal states $\left|\psi_{i}\right\rangle$.

By contrast, if the states $\left|\psi_{i}\right\rangle$ are not orthonormal then we can prove that there is no quantum measurement capable of distinguishing the states. The idea is that Bob will do a measurement described by measurement operators $M_{j}$, with outcome $j$. Depending on the outcome of the measurement Bob tries to guess what the index $i$ was using some rule, $i=f(j)$, where $f(\cdot)$ represents the rule he uses to make the guess. The key to why Bob can't distinguish non-orthogonal states $\left|\psi_{1}\right\rangle$ and $\left|\psi_{2}\right\rangle$ is the observation that $\left|\psi_{2}\right\rangle$ can be decomposed into a (non-zero) component parallel to $\left|\psi_{1}\right\rangle$, and a component orthogonal to $\left|\psi_{1}\right\rangle$. Suppose $j$ is a measurement outcome such that $f(j)=1$, that is, Bob guesses that the state was $\left|\psi_{1}\right\rangle$ when he observes $j$. But because of the component of $\left|\psi_{2}\right\rangle$ parallel\\
to $\left|\psi_{1}\right\rangle$, there is a non-zero probability of getting outcome $j$ when $\left|\psi_{2}\right\rangle$ is prepared, so sometimes Bob will make an error identifying which state was prepared. A more rigorous argument that non-orthogonal states can't be distinguished is given in Box 2.3, but this captures the essential idea.

\section*{Box 2.3: Proof that non-orthogonal states can't be reliably distinguished}
A proof by contradiction shows that no measurement distinguishing the nonorthogonal states $\left|\psi_{1}\right\rangle$ and $\left|\psi_{2}\right\rangle$ is possible. Suppose such a measurement is possible. If the state $\left|\psi_{1}\right\rangle\left(\left|\psi_{2}\right\rangle\right)$ is prepared then the probability of measuring $j$ such that $f(j)=1(f(j)=2)$ must be 1 . Defining $E_{i} \equiv \sum_{j: f(j)=i} M_{j}^{\dagger} M_{j}$, these observations may be written as:


\begin{equation*}
\left\langle\psi_{1}\left|E_{1}\right| \psi_{1}\right\rangle=1 ; \quad\left\langle\psi_{2}\left|E_{2}\right| \psi_{2}\right\rangle=1 \tag{2.99}
\end{equation*}


Since $\sum_{i} E_{i}=I$ it follows that $\sum_{i}\left\langle\psi_{1}\left|E_{i}\right| \psi_{1}\right\rangle=1$, and since $\left\langle\psi_{1}\left|E_{1}\right| \psi_{1}\right\rangle=1$ we must have $\left\langle\psi_{1}\left|E_{2}\right| \psi_{1}\right\rangle=0$, and thus $\sqrt{E_{2}}\left|\psi_{1}\right\rangle=0$. Suppose we decompose $\left|\psi_{2}\right\rangle=\alpha\left|\psi_{1}\right\rangle+\beta|\varphi\rangle$, where $|\varphi\rangle$ is orthonormal to $\left|\psi_{1}\right\rangle,|\alpha|^{2}+|\beta|^{2}=1$, and $|\beta|<1$ since $\left|\psi_{1}\right\rangle$ and $\left|\psi_{2}\right\rangle$ are not orthogonal. Then $\sqrt{E_{2}}\left|\psi_{2}\right\rangle=\beta \sqrt{E_{2}}|\varphi\rangle$, which implies a contradiction with $(2.99)$, as


\begin{equation*}
\left\langle\psi_{2}\left|E_{2}\right| \psi_{2}\right\rangle=|\beta|^{2}\left\langle\varphi\left|E_{2}\right| \varphi\right\rangle \leq|\beta|^{2}<1 \tag{2.100}
\end{equation*}


where the second last inequality follows from the observation that


\begin{equation*}
\left\langle\varphi\left|E_{2}\right| \varphi\right\rangle \leq \sum_{i}\left\langle\varphi\left|E_{i}\right| \varphi\right\rangle=\langle\varphi \mid \varphi\rangle=1 \tag{2.101}
\end{equation*}


\subsection*{2.2.5 Projective measurements}
In this section we explain an important special case of the general measurement postulate, Postulate 3. This special class of measurements is known as projective measurements. For many applications of quantum computation and quantum information we will be concerned primarily with projective measurements. Indeed, projective measurements actually turn out to be equivalent to the general measurement postulate, when they are augmented with the ability to perform unitary transformations, as described in Postulate 2. We will explain this equivalence in detail in Section 2.2.8, as the statement of the measurement postulate for projective measurements is superficially rather different from the general postulate, Postulate 3.

Projective measurements: A projective measurement is described by an observable, $M$, a Hermitian operator on the state space of the system being observed. The observable has a spectral decomposition,


\begin{equation*}
M=\sum_{m} m P_{m} \tag{2.102}
\end{equation*}


where $P_{m}$ is the projector onto the eigenspace of $M$ with eigenvalue $m$. The possible outcomes of the measurement correspond to the eigenvalues, $m$, of the observable. Upon measuring the state $|\psi\rangle$, the probability of getting result $m$ is\\
given by


\begin{equation*}
p(m)=\left\langle\psi\left|P_{m}\right| \psi\right\rangle \tag{2.103}
\end{equation*}


Given that outcome $m$ occurred, the state of the quantum system immediately after the measurement is


\begin{equation*}
\frac{P_{m}|\psi\rangle}{\sqrt{p(m)}} \tag{2.104}
\end{equation*}


Projective measurements can be understood as a special case of Postulate 3. Suppose the measurement operators in Postulate 3, in addition to satisfying the completeness relation $\sum_{m} M_{m}^{\dagger} M_{m}=I$, also satisfy the conditions that $M_{m}$ are orthogonal projectors, that is, the $M_{m}$ are Hermitian, and $M_{m} M_{m^{\prime}}=\delta_{m, m^{\prime}} M_{m}$. With these additional restrictions, Postulate 3 reduces to a projective measurement as just defined.

Projective measurements have many nice properties. In particular, it is very easy to calculate average values for projective measurements. By definition, the average (see Appendix 1 for elementary definitions and results in probability theory) value of the measurement is


\begin{align*}
\mathbf{E}(M) & =\sum_{m} m p(m)  \tag{2.110}\\
& =\sum_{m} m\left\langle\psi\left|P_{m}\right| \psi\right\rangle  \tag{2.111}\\
& =\left\langle\psi\left|\left(\sum_{m} m P_{m}\right)\right| \psi\right\rangle  \tag{2.112}\\
& =\langle\psi|M| \psi\rangle \tag{2.113}
\end{align*}


This is a useful formula, which simplifies many calculations. The average value of the observable $M$ is often written $\langle M\rangle \equiv\langle\psi|M| \psi\rangle$. From this formula for the average follows a formula for the standard deviation associated to observations of $M$,


\begin{align*}
{[\Delta(M)]^{2} } & =\left\langle(M-\langle M\rangle)^{2}\right\rangle  \tag{2.114}\\
& =\left\langle M^{2}\right\rangle-\langle M\rangle^{2} . \tag{2.115}
\end{align*}


The standard deviation is a measure of the typical spread of the observed values upon measurement of $M$. In particular, if we perform a large number of experiments in which the state $|\psi\rangle$ is prepared and the observable $M$ is measured, then the standard deviation $\Delta(M)$ of the observed values is determined by the formula $\Delta(M)=\sqrt{\left\langle M^{2}\right\rangle-\langle M\rangle^{2}}$. This formulation of measurement and standard deviations in terms of observables gives rise in an elegant way to results such as the Heisenberg uncertainty principle (see Box 2.4).

Exercise 2.58: Suppose we prepare a quantum system in an eigenstate $|\psi\rangle$ of some observable $M$, with corresponding eigenvalue $m$. What is the average observed value of $M$, and the standard deviation?

Two widely used nomenclatures for measurements deserve emphasis. Rather than giving an observable to describe a projective measurement, often people simply list a complete set of orthogonal projectors $P_{m}$ satisfying the relations $\sum_{m} P_{m}=I$ and $P_{m} P_{m^{\prime}}=$

\section*{Box 2.4: The Heisenberg uncertainty principle}
Perhaps the best known result of quantum mechanics is the Heisenberg uncertainty principle. Suppose $A$ and $B$ are two Hermitian operators, and $|\psi\rangle$ is a quantum state. Suppose $\langle\psi|A B| \psi\rangle=x+i y$, where $x$ and $y$ are real. Note that $\langle\psi|[A, B]| \psi\rangle=2 i y$ and $\langle\psi|\{A, B\}| \psi\rangle=2 x$. This implies that


\begin{equation*}
|\langle\psi|[A, B]| \psi\rangle|^{2}+|\langle\psi|\{A, B\}| \psi\rangle|^{2}=4|\langle\psi|A B| \psi\rangle|^{2} \tag{2.105}
\end{equation*}


By the Cauchy-Schwarz inequality


\begin{equation*}
|\langle\psi|A B| \psi\rangle|^{2} \leq\left\langle\psi\left|A^{2}\right| \psi\right\rangle\left\langle\psi\left|B^{2}\right| \psi\right\rangle \tag{2.106}
\end{equation*}


which combined with Equation (2.105) and dropping a non-negative term gives


\begin{equation*}
|\langle\psi|[A, B]| \psi\rangle|^{2} \leq 4\left\langle\psi\left|A^{2}\right| \psi\right\rangle\left\langle\psi\left|B^{2}\right| \psi\right\rangle \tag{2.107}
\end{equation*}


Suppose $C$ and $D$ are two observables. Substituting $A=C-\langle C\rangle$ and $B=D-\langle D\rangle$ into the last equation, we obtain Heisenberg's uncertainty principle as it is usually stated:


\begin{equation*}
\Delta(C) \Delta(D) \geq \frac{|\langle\psi|[C, D]| \psi\rangle|}{2} \tag{2.108}
\end{equation*}


You should be wary of a common misconception about the uncertainty principle, that measuring an observable $C$ to some 'accuracy' $\Delta(C)$ causes the value of $D$ to be 'disturbed' by an amount $\Delta(D)$ in such a way that some sort of inequality similar to (2.108) is satisfied. While it is true that measurements in quantum mechanics cause disturbance to the system being measured, this is most emphatically not the content of the uncertainty principle.

The correct interpretation of the uncertainty principle is that if we prepare a large number of quantum systems in identical states, $|\psi\rangle$, and then perform measurements of $C$ on some of those systems, and of $D$ in others, then the standard deviation $\Delta(C)$ of the $C$ results times the standard deviation $\Delta(D)$ of the results for $D$ will satisfy the inequality (2.108).

As an example of the uncertainty principle, consider the observables $X$ and $Y$ when measured for the quantum state $|0\rangle$. In Equation (2.70) we showed that $[X, Y]=2 i Z$, so the uncertainty principle tells us that


\begin{equation*}
\Delta(X) \Delta(Y) \geq\langle 0|Z| 0\rangle=1 \tag{2.109}
\end{equation*}


One elementary consequence of this is that $\Delta(X)$ and $\Delta(Y)$ must both be strictly greater than 0 , as can be verified by direct calculation.

$\delta_{m m^{\prime}} P_{m}$. The corresponding observable implicit in this usage is $M=\sum_{m} m P_{m}$. Another widely used phrase, to 'measure in a basis $|m\rangle$ ', where $|m\rangle$ form an orthonormal basis, simply means to perform the projective measurement with projectors $P_{m}=|m\rangle\langle m|$.

Let's look at an example of projective measurements on single qubits. First is the measurement of the observable $Z$. This has eigenvalues +1 and -1 with corresponding eigenvectors $|0\rangle$ and $|1\rangle$. Thus, for example, measurement of $Z$ on the state $|\psi\rangle=$ $(|0\rangle+|1\rangle) / \sqrt{2}$ gives the result +1 with probability $\langle\psi \mid 0\rangle\langle 0 \mid \psi\rangle=1 / 2$, and similarly the\\
result -1 with probability $1 / 2$. More generally, suppose $\vec{v}$ is any real three-dimensional unit vector. Then we can define an observable:


\begin{equation*}
\vec{v} \cdot \vec{\sigma} \equiv v_{1} \sigma_{1}+v_{2} \sigma_{2}+v_{3} \sigma_{3} \tag{2.116}
\end{equation*}


Measurement of this observable is sometimes referred to as a 'measurement of spin along the $\vec{v}$ axis', for historical reasons. The following two exercises encourage you to work out some elementary but important properties of such a measurement.

Exercise 2.59: Suppose we have qubit in the state $|0\rangle$, and we measure the observable $X$. What is the average value of $X$ ? What is the standard deviation of $X$ ?

Exercise 2.60: Show that $\vec{v} \cdot \vec{\sigma}$ has eigenvalues $\pm 1$, and that the projectors onto the corresponding eigenspaces are given by $P_{ \pm}=(I \pm \vec{v} \cdot \vec{\sigma}) / 2$.

Exercise 2.61: Calculate the probability of obtaining the result +1 for a measurement of $\vec{v} \cdot \vec{\sigma}$, given that the state prior to measurement is $|0\rangle$. What is the state of the system after the measurement if +1 is obtained?

\subsection*{2.2.6 POVM measurements}
The quantum measurement postulate, Postulate 3, involves two elements. First, it gives a rule describing the measurement statistics, that is, the respective probabilities of the different possible measurement outcomes. Second, it gives a rule describing the postmeasurement state of the system. However, for some applications the post-measurement state of the system is of little interest, with the main item of interest being the probabilities of the respective measurement outcomes. This is the case, for example, in an experiment where the system is measured only once, upon conclusion of the experiment. In such instances there is a mathematical tool known as the POVM formalism which is especially well adapted to the analysis of the measurements. (The acronym POVM stands for 'Positive Operator-Valued Measure', a technical term whose historical origins we won't worry about.) This formalism is a simple consequence of the general description of measurements introduced in Postulate 3, but the theory of POVMs is so elegant and widely used that it merits a separate discussion here.

Suppose a measurement described by measurement operators $M_{m}$ is performed upon a quantum system in the state $|\psi\rangle$. Then the probability of outcome $m$ is given by $p(m)=\left\langle\psi\left|M_{m}^{\dagger} M_{m}\right| \psi\right\rangle$. Suppose we define


\begin{equation*}
E_{m} \equiv M_{m}^{\dagger} M_{m} \tag{2.117}
\end{equation*}


Then from Postulate 3 and elementary linear algebra, $E_{m}$ is a positive operator such that $\sum_{m} E_{m}=I$ and $p(m)=\left\langle\psi\left|E_{m}\right| \psi\right\rangle$. Thus the set of operators $E_{m}$ are sufficient to determine the probabilities of the different measurement outcomes. The operators $E_{m}$ are known as the POVM elements associated with the measurement. The complete set $\left\{E_{m}\right\}$ is known as a $P O V M$.

As an example of a POVM, consider a projective measurement described by measurement operators $P_{m}$, where the $P_{m}$ are projectors such that $P_{m} P_{m^{\prime}}=\delta_{m m^{\prime}} P_{m}$ and $\sum_{m} P_{m}=I$. In this instance (and only this instance) all the POVM elements are the same as the measurement operators themselves, since $E_{m} \equiv P_{m}^{\dagger} P_{m}=P_{m}$.

Box 2.5: General measurements, projective measurements, and POVMs

Most introductions to quantum mechanics describe only projective measurements, and consequently the general description of measurements given in Postulate 3 may be unfamiliar to many physicists, as may the POVM formalism described in Section 2.2.6. The reason most physicists don't learn the general measurement formalism is because most physical systems can only be measured in a very coarse manner. In quantum computation and quantum information we aim for an exquisite level of control over the measurements that may be done, and consequently it helps to use a more comprehensive formalism for the description of measurements.

Of course, when the other axioms of quantum mechanics are taken into account, projective measurements augmented by unitary operations turn out to be completely equivalent to general measurements, as shown in Section 2.2.8. So a physicist trained in the use of projective measurements might ask to what end we start with the general formalism, Postulate 3? There are several reasons for doing so. First, mathematically general measurements are in some sense simpler than projective measurements, since they involve fewer restrictions on the measurement operators; there is, for example, no requirement for general measurements analogous to the condition $P_{i} P_{j}=\delta_{i j} P_{i}$ for projective measurements. This simpler structure also gives rise to many useful properties for general measurements that are not possessed by projective measurements. Second, it turns out that there are important problems in quantum computation and quantum information - such as the optimal way to distinguish a set of quantum states - the answer to which involves a general measurement, rather than a projective measurement.

A third reason for preferring Postulate 3 as a starting point is related to a property of projective measurements known as repeatability. Projective measurements are repeatable in the sense that if we perform a projective measurement once, and obtain the outcome $m$, repeating the measurement gives the outcome $m$ again and does not change the state. To see this, suppose $|\psi\rangle$ was the initial state. After the first measurement the state is $\left|\psi_{m}\right\rangle=\left(P_{m}|\psi\rangle\right) / \sqrt{\left\langle\psi\left|P_{m}\right| \psi\right\rangle}$. Applying $P_{m}$ to $\left|\psi_{m}\right\rangle$ does not change it, so we have $\left\langle\psi_{m}\left|P_{m}\right| \psi_{m}\right\rangle=1$, and therefore repeated measurement gives the result $m$ each time, without changing the state.

This repeatability of projective measurements tips us off to the fact that many important measurements in quantum mechanics are not projective measurements. For instance, if we use a silvered screen to measure the position of a photon we destroy the photon in the process. This certainly makes it impossible to repeat the measurement of the photon's position! Many other quantum measurements are also not repeatable in the same sense as a projective measurement. For such measurements, the general measurement postulate, Postulate 3 , must be employed. Where do POVMs fit in this picture? POVMs are best viewed as a special case of the general measurement formalism, providing the simplest means by which one can study general measurement statistics, without the necessity for knowing the post-measurement state. They are a mathematical convenience that sometimes gives extra insight into quantum measurements.

Exercise 2.62: Show that any measurement where the measurement operators and the POVM elements coincide is a projective measurement.

Above we noticed that the POVM operators are positive and satisfy $\sum_{m} E_{m}=I$. Suppose now that $\left\{E_{m}\right\}$ is some arbitrary set of positive operators such that $\sum_{m} E_{m}=I$. We will show that there exists a set of measurement operators $M_{m}$ defining a measurement described by the POVM $\left\{E_{m}\right\}$. Defining $M_{m} \equiv \sqrt{E_{m}}$ we see that $\sum_{m} M_{m}^{\dagger} M_{m}=$ $\sum_{m} E_{m}=I$, and therefore the set $\left\{M_{m}\right\}$ describes a measurement with POVM $\left\{E_{m}\right\}$. For this reason it is convenient to define a POVM to be any set of operators $\left\{E_{m}\right\}$ such that: (a) each operator $E_{m}$ is positive; and (b) the completeness relation $\sum_{m} E_{m}=I$ is obeyed, expressing the fact that probabilities sum to one. To complete the description of POVMs, we note again that given a POVM $\left\{E_{m}\right\}$, the probability of outcome $m$ is given by $p(m)=\left\langle\psi\left|E_{m}\right| \psi\right\rangle$.

We've looked at projective measurements as an example of the use of POVMs, but it wasn't very exciting since we didn't learn much that was new. The following more sophisticated example illustrates the use of the POVM formalism as a guide for our intuition in quantum computation and quantum information. Suppose Alice gives Bob a qubit prepared in one of two states, $\left|\psi_{1}\right\rangle=|0\rangle$ or $\left|\psi_{2}\right\rangle=(|0\rangle+|1\rangle) / \sqrt{2}$. As explained in Section 2.2.4 it is impossible for Bob to determine whether he has been given $\left|\psi_{1}\right\rangle$ or $\left|\psi_{2}\right\rangle$ with perfect reliability. However, it is possible for him to perform a measurement which distinguishes the states some of the time, but never makes an error of mis-identification. Consider a POVM containing three elements,


\begin{align*}
& E_{1} \equiv \frac{\sqrt{2}}{1+\sqrt{2}}|1\rangle\langle 1|  \tag{2.118}\\
& E_{2} \equiv \frac{\sqrt{2}}{1+\sqrt{2}} \frac{(|0\rangle-|1\rangle)(\langle 0|-\langle 1|)}{2}  \tag{2.119}\\
& E_{3} \equiv I-E_{1}-E_{2} \tag{2.120}
\end{align*}


It is straightforward to verify that these are positive operators which satisfy the completeness relation $\sum_{m} E_{m}=I$, and therefore form a legitimate POVM.

Suppose Bob is given the state $\left|\psi_{1}\right\rangle=|0\rangle$. He performs the measurement described by the POVM $\left\{E_{1}, E_{2}, E_{3}\right\}$. There is zero probability that he will observe the result $E_{1}$, since $E_{1}$ has been cleverly chosen to ensure that $\left\langle\psi_{1}\left|E_{1}\right| \psi_{1}\right\rangle=0$. Therefore, if the result of his measurement is $E_{1}$ then Bob can safely conclude that the state he received must have been $\left|\psi_{2}\right\rangle$. A similar line of reasoning shows that if the measurement outcome $E_{2}$ occurs then it must have been the state $\left|\psi_{1}\right\rangle$ that Bob received. Some of the time, however, Bob will obtain the measurement outcome $E_{3}$, and he can infer nothing about the identity of the state he was given. The key point, however, is that Bob never makes a mistake identifying the state he has been given. This infallibility comes at the price that sometimes Bob obtains no information about the identity of the state.

This simple example demonstrates the utility of the POVM formalism as a simple and intuitive way of gaining insight into quantum measurements in instances where only the measurement statistics matter. In many instances later in the book we will only be concerned with measurement statistics, and will therefore use the POVM formalism rather than the more general formalism for measurements described in Postulate 3.

Exercise 2.63: Suppose a measurement is described by measurement operators $M_{m}$.

Show that there exist unitary operators $U_{m}$ such that $M_{m}=U_{m} \sqrt{E_{m}}$, where $E_{m}$ is the POVM associated to the measurement.

Exercise 2.64: Suppose Bob is given a quantum state chosen from a set $\left|\psi_{1}\right\rangle, \ldots,\left|\psi_{m}\right\rangle$ of linearly independent states. Construct a $\operatorname{POVM}\left\{E_{1}, E_{2}, \ldots, E_{m+1}\right\}$ such that if outcome $E_{i}$ occurs, $1 \leq i \leq m$, then Bob knows with certainty that he was given the state $\left|\psi_{i}\right\rangle$. (The POVM must be such that $\left\langle\psi_{i}\left|E_{i}\right| \psi_{i}\right\rangle>0$ for each $i$.)

\subsection*{2.2.7 Phase}
'Phase' is a commonly used term in quantum mechanics, with several different meanings dependent upon context. At this point it is convenient to review a couple of these meanings. Consider, for example, the state $e^{i \theta}|\psi\rangle$, where $|\psi\rangle$ is a state vector, and $\theta$ is a real number. We say that the state $e^{i \theta}|\psi\rangle$ is equal to $|\psi\rangle$, up to the global phase factor $e^{i \theta}$. It is interesting to note that the statistics of measurement predicted for these two states are the same. To see this, suppose $M_{m}$ is a measurement operator associated to some quantum measurement, and note that the respective probabilities for outcome $m$ occurring are $\left\langle\psi\left|M_{m}^{\dagger} M_{m}\right| \psi\right\rangle$ and $\left\langle\psi\left|e^{-i \theta} M_{m}^{\dagger} M_{m} e^{i \theta}\right| \psi\right\rangle=\left\langle\psi\left|M_{m}^{\dagger} M_{m}\right| \psi\right\rangle$. Therefore, from an observational point of view these two states are identical. For this reason we may ignore global phase factors as being irrelevant to the observed properties of the physical system.

There is another kind of phase known as the relative phase, which has quite a different meaning. Consider the states


\begin{equation*}
\frac{|0\rangle+|1\rangle}{\sqrt{2}} \quad \text { and } \quad \frac{|0\rangle-|1\rangle}{\sqrt{2}} \text {. } \tag{2.121}
\end{equation*}


In the first state the amplitude of $|1\rangle$ is $1 / \sqrt{2}$. For the second state the amplitude is $-1 / \sqrt{2}$. In each case the magnitude of the amplitudes is the same, but they differ in sign. More generally, we say that two amplitudes, $a$ and $b$, differ by a relative phase if there is a real $\theta$ such that $a=\exp (i \theta) b$. More generally still, two states are said to differ by a relative phase in some basis if each of the amplitudes in that basis is related by such a phase factor. For example, the two states displayed above are the same up to a relative phase shift because the $|0\rangle$ amplitudes are identical (a relative phase factor of 1 ), and the $|1\rangle$ amplitudes differ only by a relative phase factor of -1 . The difference between relative phase factors and global phase factors is that for relative phase the phase factors may vary from amplitude to amplitude. This makes the relative phase a basis-dependent concept unlike global phase. As a result, states which differ only by relative phases in some basis give rise to physically observable differences in measurement statistics, and it is not possible to regard these states as physically equivalent, as we do with states differing by a global phase factor

Exercise 2.65: Express the states $(|0\rangle+|1\rangle) / \sqrt{2}$ and $(|0\rangle-|1\rangle) / \sqrt{2}$ in a basis in which they are not the same up to a relative phase shift.

\subsection*{2.2.8 Composite systems}
Suppose we are interested in a composite quantum system made up of two (or more) distinct physical systems. How should we describe states of the composite system? The following postulate describes how the state space of a composite system is built up from the state spaces of the component systems.

Postulate 4: The state space of a composite physical system is the tensor product of the state spaces of the component physical systems. Moreover, if we have systems numbered 1 through $n$, and system number $i$ is prepared in the state $\left|\psi_{i}\right\rangle$, then the joint state of the total system is $\left|\psi_{1}\right\rangle \otimes\left|\psi_{2}\right\rangle \otimes \cdots \otimes\left|\psi_{n}\right\rangle$.

Why is the tensor product the mathematical structure used to describe the state space of a composite physical system? At one level, we can simply accept it as a basic postulate, not reducible to something more elementary, and move on. After all, we certainly expect that there be some canonical way of describing composite systems in quantum mechanics. Is there some other way we can arrive at this postulate? Here is one heuristic that is sometimes used. Physicists sometimes like to speak of the superposition principle of quantum mechanics, which states that if $|x\rangle$ and $|y\rangle$ are two states of a quantum system, then any superposition $\alpha|x\rangle+\beta|y\rangle$ should also be an allowed state of a quantum system, where $|\alpha|^{2}+|\beta|^{2}=1$. For composite systems, it seems natural that if $|A\rangle$ is a state of system $A$, and $|B\rangle$ is a state of system $B$, then there should be some corresponding state, which we might denote $|A\rangle|B\rangle$, of the joint system $A B$. Applying the superposition principle to product states of this form, we arrive at the tensor product postulate given above. This is not a derivation, since we are not taking the superposition principle as a fundamental part of our description of quantum mechanics, but it gives you the flavor of the various ways in which these ideas are sometimes reformulated.

A variety of different notations for composite systems appear in the literature. Part of the reason for this proliferation is that different notations are better adapted for different applications, and we will also find it convenient to introduce some specialized notations on occasion. At this point it suffices to mention a useful subscript notation to denote states and operators on different systems, when it is not clear from context. For example, in a system containing three qubits, $X_{2}$ is the Pauli $\sigma_{x}$ operator acting on the second qubit.

Exercise 2.66: Show that the average value of the observable $X_{1} Z_{2}$ for a two qubit system measured in the state $(|00\rangle+|11\rangle) / \sqrt{2}$ is zero.

In Section 2.2 .5 we claimed that projective measurements together with unitary dynamics are sufficient to implement a general measurement. The proof of this statement makes use of composite quantum systems, and is a nice illustration of Postulate 4 in action. Suppose we have a quantum system with state space $Q$, and we want to perform a measurement described by measurement operators $M_{m}$ on the system $Q$. To do this, we introduce an ancilla system, with state space $M$, having an orthonormal basis $|m\rangle$ in one-to-one correspondence with the possible outcomes of the measurement we wish to implement. This ancilla system can be regarded as merely a mathematical device appearing in the construction, or it can be interpreted physically as an extra quantum system introduced into the problem, which we assume has a state space with the required properties.

Letting $|0\rangle$ be any fixed state of $M$, define an operator $U$ on products $|\psi\rangle|0\rangle$ of states $|\psi\rangle$ from $Q$ with the state $|0\rangle$ by


\begin{equation*}
U|\psi\rangle|0\rangle \equiv \sum_{m} M_{m}|\psi\rangle|m\rangle \tag{2.122}
\end{equation*}


Using the orthonormality of the states $|m\rangle$ and the completeness relation $\sum_{m} M_{m}^{\dagger} M_{m}=$\\
$I$, we can see that $U$ preserves inner products between states of the form $|\psi\rangle|0\rangle$,


\begin{align*}
\left\langle\varphi\left|\left\langle 0\left|U^{\dagger} U\right| \psi\right\rangle\right| 0\right\rangle & =\sum_{m, m^{\prime}}\left\langle\varphi\left|M_{m}^{\dagger} M_{m^{\prime}}\right| \psi\right\rangle\left\langle m \mid m^{\prime}\right\rangle  \tag{2.123}\\
& =\sum_{m}\left\langle\varphi\left|M_{m}^{\dagger} M_{m}\right| \psi\right\rangle  \tag{2.124}\\
& =\langle\varphi \mid \psi\rangle . \tag{2.125}
\end{align*}


By the results of Exercise 2.67 it follows that $U$ can be extended to a unitary operator on the space $Q \otimes M$, which we also denote by $U$.

Exercise 2.67: Suppose $V$ is a Hilbert space with a subspace $W$. Suppose

$U: W \rightarrow V$ is a linear operator which preserves inner products, that is, for any $\left|w_{1}\right\rangle$ and $\left|w_{2}\right\rangle$ in $W$


\begin{equation*}
\left\langle w_{1}\left|U^{\dagger} U\right| w_{2}\right\rangle=\left\langle w_{1} \mid w_{2}\right\rangle \tag{2.126}
\end{equation*}


Prove that there exists a unitary operator $U^{\prime}: V \rightarrow V$ which extends $U$. That is, $U^{\prime}|w\rangle=U|w\rangle$ for all $|w\rangle$ in $W$, but $U^{\prime}$ is defined on the entire space $V$. Usually we omit the prime symbol ' and just write $U$ to denote the extension.

Next, suppose we perform a projective measurement on the two systems described by projectors $P_{m} \equiv I_{Q} \otimes|m\rangle\langle m|$. Outcome $m$ occurs with probability


\begin{align*}
p(m) & =\left\langle\psi\left|\left\langle 0\left|U^{\dagger} P_{m} U\right| \psi\right\rangle\right| 0\right\rangle  \tag{2.127}\\
& =\sum_{m^{\prime}, m^{\prime \prime}}\left\langle\psi\left|M_{m^{\prime}}^{\dagger}\left\langle m^{\prime}\left|\left(I_{Q} \otimes|m\rangle\langle m|\right) M_{m^{\prime \prime}}\right| \psi\right\rangle\right| m^{\prime \prime}\right\rangle  \tag{2.128}\\
& =\left\langle\psi\left|M_{m}^{\dagger} M_{m}\right| \psi\right\rangle \tag{2.129}
\end{align*}


just as given in Postulate 3. The joint state of the system $Q M$ after measurement, conditional on result $m$ occurring, is given by


\begin{equation*}
\frac{P_{m} U|\psi\rangle|0\rangle}{\sqrt{\left\langle\psi\left|U^{\dagger} P_{m} U\right| \psi\right\rangle}}=\frac{M_{m}|\psi\rangle|m\rangle}{\sqrt{\left\langle\psi\left|M_{m}^{\dagger} M_{m}\right| \psi\right\rangle}} \tag{2.130}
\end{equation*}


It follows that the state of system $M$ after the measurement is $|m\rangle$, and the state of system $Q$ is


\begin{equation*}
\frac{M_{m}|\psi\rangle}{\sqrt{\left\langle\psi\left|M_{m}^{\dagger} M_{m}\right| \psi\right\rangle}} \tag{2.131}
\end{equation*}


just as prescribed by Postulate 3. Thus unitary dynamics, projective measurements, and the ability to introduce ancillary systems, together allow any measurement of the form described in Postulate 3 to be realized.

Postulate 4 also enables us to define one of the most interesting and puzzling ideas associated with composite quantum systems - entanglement. Consider the two qubit state


\begin{equation*}
|\psi\rangle=\frac{|00\rangle+|11\rangle}{\sqrt{2}} . \tag{2.132}
\end{equation*}


This state has the remarkable property that there are no single qubit states $|a\rangle$ and $|b\rangle$ such that $|\psi\rangle=|a\rangle|b\rangle$, a fact which you should now convince yourself of:

Exercise 2.68: Prove that $|\psi\rangle \neq|a\rangle|b\rangle$ for all single qubit states $|a\rangle$ and $|b\rangle$.

We say that a state of a composite system having this property (that it can't be written as a product of states of its component systems) is an entangled state. For reasons which nobody fully understands, entangled states play a crucial role in quantum computation and quantum information, and arise repeatedly through the remainder of this book. We have already seen entanglement play a crucial role in quantum teleportation, as described in Section 1.3.7. In this chapter we give two examples of the strange effects enabled by entangled quantum states, superdense coding (Section 2.3), and the violation of Bell's inequality (Section 2.6).

\subsection*{2.2.9 Quantum mechanics: a global view}
We have now explained all the fundamental postulates of quantum mechanics. Most of the rest of the book is taken up with deriving consequences of these postulates. Let's quickly review the postulates and try to place them in some kind of global perspective.

Postulate 1 sets the arena for quantum mechanics, by specifying how the state of an isolated quantum system is to be described. Postulate 2 tells us that the dynamics of closed quantum systems are described by the Schrödinger equation, and thus by unitary evolution. Postulate 3 tells us how to extract information from our quantum systems by giving a prescription for the description of measurement. Postulate 4 tells us how the state spaces of different quantum systems may be combined to give a description of the composite system.

What's odd about quantum mechanics, at least by our classical lights, is that we can't directly observe the state vector. It's a little bit like a game of chess where you can never find out exactly where each piece is, but only know the rank of the board they are on. Classical physics - and our intuition - tells us that the fundamental properties of an object, like energy, position, and velocity, are directly accessible to observation. In quantum mechanics these quantities no longer appear as fundamental, being replaced by the state vector, which can't be directly observed. It is as though there is a hidden world in quantum mechanics, which we can only indirectly and imperfectly access. Moreover, merely observing a classical system does not necessarily change the state of the system. Imagine how difficult it would be to play tennis if each time you looked at the ball its position changed! But according to Postulate 3, observation in quantum mechanics is an invasive procedure that typically changes the state of the system.

What conclusions should we draw from these strange features of quantum mechanics? Might it be possible to reformulate quantum mechanics in a mathematically equivalent way so that it had a structure more like classical physics? In Section 2.6 we'll prove Bell's inequality, a surprising result that shows any attempt at such a reformulation is doomed to failure. We're stuck with the counter-intuitive nature of quantum mechanics. Of course, the proper reaction to this is glee, not sorrow! It gives us an opportunity to develop tools of thought that make quantum mechanics intuitive. Moreover, we can exploit the hidden nature of the state vector to do information processing tasks beyond what is possible in the classical world. Without this counter-intuitive behavior, quantum computation and quantum information would be a lot less interesting.

We can also turn this discussion about, and ask ourselves: 'If quantum mechanics is so different from classical physics, then how come the everyday world looks so classical?' Why do we see no evidence of a hidden state vector in our everyday lives? It turns out


\end{document}